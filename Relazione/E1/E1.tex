\Chapter{Esercitazione 1}{Velivolo convenzionale}
\label{sec:E1}

Nella prima esercitazione si è progettato un nuovo velivolo convenzionale, di tipo trasporto passeggeri a medio raggio, come possibile successore dell'Airbus A320.

\section{Requisito}
Per definire il requisito del nuovo velivolo si è partiti innanzitutto dall'analisi statistica degli aerei commerciali dello stesso settore di maggior diffusione sul mercato odierno.

\begin{table}[H]
\centering
\footnotesize
\begin{tabular}{ccccccccccccc}
Model&Max&Length&Wing&Fuselage&Wing&Aspect&MTOW&Max&Fuel&OEW&Cruise&Range\\
&Seats&&Span&Length&Area&Ratio&&Payload&&&Speed&\\
&&[m]&[m]&[m]&$\text{[m}^2\text{]}$&&[t]&[t]&[l]&[t]&[km/h]&[km]\\
\toprule
\normalsize\textbf{Airbus}&&&&&&&&&&&& \\
\midrule
A220-300 & 160 & 38.71 & 35.1 & 3.5 & 112.3 & 8.89 & 70.9 & 18.7 & 21920 & 37.1 & 823 & 6300 \\
A318 & 136 & 31.45 & 34.1 & 3.95 & 122.4 & 7.43 & 59-68 & 15 & 24210 & 39.6 & 829 & 5700 \\
A319ceo & 160 & 33.84 & 34.1 & 3.95 & 122.6 & 7.41 & 75.5 & 17.7 & 24210 & 40.8 & 829 & 6850 \\
A319neo & 160 & 33.84 & 35.8 & 3.95 & 124 & 8.18 & 78.2 & 17.7 & 24210 & 42.6 & 833 & 6950 \\
A320ceo & 180 & 37.57 & 34.1 & 3.95 & 122.6 & 7.41 & 78 & 19.9 & 24210 & 42.6 & 829 & 6100 \\
\textbf{A320neo} & \textbf{195} & \textbf{37.57} & \textbf{35.8} & \textbf{3.95} & \textbf{124} & \textbf{8.18} & \textbf{79} & \textbf{20} & \textbf{24210} & \textbf{44.3} & \textbf{833} & \textbf{6500} \\
A321ceo & 220 & 44.51 & 34.1 & 3.95 & 122.6 & 7.41 & 89 & 22.8 & 23700 & 48 & 829 & 5900 \\
A321neo & 244 & 44.51 & 35.8 & 3.95 & 124 & 8.18 & 93.5 & 25.5 & 23500 & 50.1 & 833 & 6500 \\
A321XLR & 244 & 44.51 & 35.8 & 3.95 & 124 & 8.18 & 101 & 25.5 & 40000 & 50.1 & 833 & 8700 \\
\toprule
\normalsize\textbf{Boeing}&&&&&&&&&&&& \\
\midrule
B737-700 & 149 & 32.18 & 34.3 & 3.73 & 124.6 & 7.50 & 69.4 & 17 & 26000 & 37.6 & 838 & 6000 \\
B737-800 & 189 & 38.08 & 34.3 & 3.73 & 124.6 & 7.50 & 78.2 & 20.2 & 26000 & 41.4 & 838 & 5420 \\
B737-7 & 172 & 35.56 & 35.92 & 3.76 & 127 & 8.14 & 80 & -- & 25800 & -- & 839 & 7000 \\
B737-8 & 189 & 39.47 & 35.92 & 3.76 & 127 & 8.14 & 82.6 & -- & 25800 & -- & 839 & 6500 \\
B737-9 & 220 & 42.16 & 35.92 & 3.76 & 127 & 8.14 & 88.3 & -- & 25800 & -- & 839 & 6100 \\
B737-10 & 230 & 43.79 & 35.92 & 3.76 & 127 & 8.14 & 89.8 & -- & 25800 & -- & 839 & 5700 \\
\toprule
\normalsize\textbf{Comac}&&&&&&&&&&&& \\
\midrule
C919 & 192 & 38.9 & 35.8 & 3.96 & 129.1 & 7.85 & 75.1 & 18.9 & 24900 & 45.7 & 838 & 4140 \\
C919ER & 192 & 38.9 & 35.8 & 3.96 & 129.1 & 7.85 & 78.9 & 18.9 & 24900 & 45.7 & 838 & 5576 \\
\bottomrule
\end{tabular}
\caption{Dati di aerei dello stesso segmento.}
\end{table}

\noindent A partire dalla ricerca di mercato, si sono stilati una serie di requisiti per il nuovo velivolo.

\begin{table}[H]
\centering
\begin{tabular}{c|c}
Numero di passeggeri & 180 \\
Numero di piloti & 2 \\
Assistenti di volo & 4 \\
Velocità di crociera & 835 km/h \\
Range & 4500 km \\
Range di diversione & 200 nm \\
Numero di motori & 2 \\
Specific Fuel Consumption & 0.5 $kg/(kg\cdot h)$ \\
\end{tabular}
\caption{Requisiti del nuovo velivolo.}
\end{table}

Il numero di assistenti di volo è stato determinato a partire dal numero dei passeggeri, con la normativa EASA ORO.CC.100(b)(3) (un assistente ogni 50 passeggeri); lo Specific Fuel Consumption è leggermente migliorato rispetto all'Airbus A320neo, ipotizzando un avanzamento tecnologico che permette di migliorare l'efficienza dei propulsori.
\vspace{0.5cm}
\newline Un altro fondamentale requisito vincolante è che l'apertura alare sia compresa tra i 24 e i 36 metri, in modo da far rientrare il velivolo nella classe C della classificazione dei velivoli definita dall'ICAO.
\newpage
\section{Metodologia}
\vspace{1cm}
\subsection{Stima preliminare della massa massima al decollo}
Il primo parametro che si è stimato, fondamentale per il resto del progetto, è la massa massima al decollo. È possibile dividerla in quattro componenti: equipaggio, payload, carburante e massa a vuoto. 
\[m_{TO} = m_{crew}+m_{payload}+m_{fuel}+m_{empty}\]
È possibile riarrangiare la precedente equazione in modo da esplicitare la frazione di carburante e la frazione di massa a vuoto
\[m_{TO} = \dfrac{m_{crew}+m_{payload}}{1-\dfrac{m_{fuel}}{m_{TO}}-\dfrac{m_{empty}}{m_{TO}}}\]
Si sono quindi calcolate le diverse componenti della massa.

\subsubsection{Massa equipaggio e payload}
In questa fase iniziale di progetto si è assunto che tutto il payload sia fatto dai passeggeri e dai loro bagagli, non considerando quindi eventuali merci aggiuntive.

Si è quindi calcolata la massa considerando la stima statistica elaborata da Roskam, secondo cui il peso di un singolo passeggero è 79.4 kg e il suo bagaglio, per un volo di corto-medio raggio, 13.6 kg, per un totale di \textbf{93.0 kg}. Si è assunto lo stesso peso per il personale di bordo.

In definitiva, la massa totale per le 186 persone a bordo (180 passeggeri + 6 crew) è pari a \textbf{17298 kg}.

\subsubsection{Frazione di massa a vuoto}
È stata utilizzata la stima di Raymer, secondo cui
\[ \dfrac{m_{empty}}{m_{TO}} = A*m_{TO}^C\]
Dove \textit{A} e \textit{C} sono parametri costanti, dipendenti dal tipo di velivolo. Per un aereo da trasporto a getto valgono rispettivamente 0.97 e -0.06.

Si nota che questa frazione dipende dalla stessa massa al decollo; sarà quindi necessario effettuare un calcolo iterativo.

\subsubsection{Frazione di carburante}
La quantità di carburante necessaria può essere stimata a partire dalla missione tipo del velivolo. 
\begin{figure}[H]
    \centering
    \includegraphics[width=0.7\linewidth]{E1//Immagini/missione.png}
    \caption{Missione tipica di un aereo commerciale.}
\end{figure}

È quindi possibile dividere la missione in più fasi, e per ogni fase stimare il rapporto di peso tra la fine e l'inizio della fase stessa. 

In particolare è possibile ricavare questa stima da dati statistici per tutte le fasi tranne tre, ovvero la crociera, la diversione (che si tratta a tutti gli effetti di una seconda crociera) e il loiter. Per queste tre fasi è necessario ricorrere alle equazioni di autonomia (chilometrica e oraria) ricavabili dalla meccanica del volo:
\begin{align}
\text{Autonomia chilometrica: }&\dfrac{W_{i+1}}{W_i} = e^{-\dfrac{c\Delta x}{VE}}
\label{eq:autonomia} \\
\text{Autonomia oraria: }&\dfrac{W_{i+1}}{W_i} = e^{-\dfrac{c\Delta t}{E}} 
\end{align}
dove c è lo \textit{Specific Fuel Consumption}, V la velocità ed E l'efficienza aerodinamica. In particolare per lo SFC del loiter si è usato il valore di 0.4 $kg/(kg\cdot h)$, inferiore rispetto allo 0.5 $kg/(kg\cdot h)$ della crociera. Inoltre, il range considerato in questo step è di 4500 km, migliorando di 500 km quello dell'A320neo a pieno carico.

Queste formule sono ricavate facendo una semplificazione: quella di assetto costante, quindi $C_L$ costante; ne consegue che durante la crociera l'aereo prende quota grazie alla riduzione di peso dovuta al consumo di carburante, per una condizione di cosiddetto \textit{cruise-climb}.

È stato quindi necessario ricavare il valore dell'efficienza: si è utilizzata la formula, di nuovo ricavata da Raymer, secondo cui
\[\left(\dfrac{L}{D}\right)_{max} = k_E \sqrt{AR_w}\]
dove $k_E$ è un coefficiente dipendente dal tipo di aeroplano, in questo caso pari a 15.5, ed $AR_w$ l'aspect ratio della superficie bagnata:
\[AR_w = \dfrac{AR}{\dfrac{S_{wet}}{S_{ref}}}\]
dove AR è l'aspect ratio dell'ala, in questa fase preliminare imposta a 9 (in linea con aerei già esistenti) e il rapporto di superfici al denominatore è ricavato da aerei simili e posto pari a 6.4.

Si è quindi ricavata un'efficienza massima pari a 18.4, realistica per un'aereo molto efficiente. Questa è pari all'efficienza nella fase di loiter, mentre per la fase di crociera va ridotta di un fattore 0.866.

Si sono infine calcolati i rapporti di peso delle tre fasi:
\begin{table}[H]
\centering
\begin{tabular}{ccc}
$\left(\dfrac{W_f}{W_i}\right)_{cruise} = 0.8443$ ; & $\left(\dfrac{W_f}{W_i}\right)_{diversion} = 0.974$ ; & $\left(\dfrac{W_f}{W_i}\right)_{loiter} = 0.989$ .
\end{tabular}
\end{table}

Come anticipato, il rapporto di peso delle altre fasi si ricava da dati statistici:
\begin{table}[H]
\centering
\begin{tabular}{ccc}
$\left(\dfrac{W_f}{W_i}\right)_{take off} = 0.970$ ; & $\left(\dfrac{W_f}{W_i}\right)_{climb} = 0.985$ ; & $\left(\dfrac{W_f}{W_i}\right)_{descent} = 0.995$ ;
\end{tabular}
\end{table}
\begin{table}[H]
\centering
\begin{tabular}{cc}
$\left(\dfrac{W_f}{W_i}\right)_{diversion\,climb} = 1$ ; & $\left(\dfrac{W_f}{W_i}\right)_{diversion\,descent} = 1$ .
\end{tabular}
\end{table}

Avendo infine ricavato il rapporto di pesi per tutte le fasi, è possibile calcolare il rapporto di peso tra la fine e l'inizio della missione:

\begin{equation}
    \left(\dfrac{W_f}{W_i}\right)_{mission} = \dfrac{W_{final}}{W_{TO}} = \prod_{i=1}^{n}\left(\dfrac{W_{i+1}}{W_i}\right) = 0.7734
    \label{eq:w_final}
\end{equation}

Il complemento ad uno di questo valore è la frazione di carburante dell'aereo. A questo valore va aggiunto il 5\% di riserva:
\begin{equation}
    \frac{m_{fuel}}{m_{TO}} = 1.05\left(1-\dfrac{m_{final}}{m_{TO}}\right) = \textbf{0.238}
    \label{eq:frazione_carburante}
\end{equation}

\subsubsection{Peso al decollo}
Ricavati tutti gli elementi, è stato infine calcolato il peso al decollo tramite un calcolo iterativo su MATLAB. Il codice converge dopo 6 iterazioni, a partire da una stima iniziale di 60 tonnellate e con una tolleranza sulla differenza relativa di $10^{-4}$, su un valore di \textbf{65657 kg}. È un valore sostanzialmente inferiore di quello dell'A320neo di riferimento, giustificato dal minor numero di passeggeri e dalla mancanza di trasporto merci.

\subsubsection{Analisi di trade-off}
A partire da questo risultato, sono state effettuate analisi di trade-off al variare di alcuni parametri.

\begin{table}[H]
    \centering
    \begin{tabular}{c|c}
    \toprule
    Parametro & Range di valori \\
    \midrule
        Range & 3500 - 5500 km \\
        Payload &  160 - 200 passeggeri\\
        Specific Fuel Consumption & 0.4 - 0.6 $kg/(kg\cdot h)$ \\
        Aspect Ratio & 8 - 10 \\
         \bottomrule
    \end{tabular}
    \caption{Parametri per l'analisi di trade-off}
\end{table}

I valori ottenuti sono illustrati nei plot seguenti.

\begin{figure}[H]
    \centering
    \begin{subfigure}[b]{0.48\textwidth}
        \includegraphics[width=\textwidth]{E1/Immagini/tradeoff_range.png}
        \caption{Massa al decollo al variare del range}
    \end{subfigure}
    \hfill
    \begin{subfigure}[b]{0.48\textwidth}
        \includegraphics[width=\textwidth]{E1/Immagini/tradeoff_payload.png}
        \caption{Massa al decollo al variare del payload}
    \end{subfigure}
    \medskip
    \begin{subfigure}[b]{0.48\textwidth}
        \includegraphics[width=\textwidth]{E1/Immagini/tradeoff_sfc.png}
        \caption{Massa al decollo al variare dello SFC}
    \end{subfigure}
    \hfill
    \begin{subfigure}[b]{0.48\textwidth}
        \includegraphics[width=\textwidth]{E1/Immagini/tradeoff_AR.png}
        \caption{Massa al decollo al variare dell'aspect ratio}
    \end{subfigure}
    \medskip
    \caption{Risultato delle analisi di trade-off}
\end{figure}

Come previsto la massa aumenta con l'aumentare del range, del payload e del consumo, mentre diminuisce all'aumentare dell'aspect ratio in quanto questo parametro è direttamente collegato all'efficienza aerodinamica.

\vspace{1cm}
\subsection{Diagramma Payload-Range preliminare}
Basandosi sui dati ricavati dal calcolo delle masse, si è ricavato il diagramma payload-range per l'aeroplano. Esso è definito da quattro punti:
\begin{itemize}
    \item \textbf{Punto A}: corrisponde al payload massimo e senza carburante a bordo;
    \item \textbf{Punto B}: corrisponde al \textit{punto armonico}, ovvero il range massimo a payload massimo. È il punto di progetto dell'aeroplano, quindi con range e payload definiti da requisito;
    \item \textbf{Punto C}: corrisponde al range massimo raggiungibile con pieno carico di carburante, sacrificando del payload per rientrare nel peso massimo al decollo;
    \item \textbf{Punto D}: corrisponde al range massimo teorico raggiungibile dall'aeroplano vuoto, ovvero senza payload e con carburante massimo.
\end{itemize}

\noindent Il primo punto si ricava banalmente, il range è di 0 km ed il payload quello massimo dei passeggeri.

\noindent Il secondo punto è ugualmente semplice da ricavare, in quanto corrisponde al payload massimo ed il range di requisito.

\medskip

\noindent Per calcolare il terzo punto si sono dovute fare delle assunzioni. Innanzitutto la capacità dei serbatoi, fissata a \textbf{25000 l} in base alla capacità di aerei della stessa categoria; inoltre è stato necessario assumere la densità del carburante: si è utilizzato il valore di \textbf{0.785 kg/l}, coerente con la letteratura per il carburante \textit{Jet-A} e lo stesso valore utilizzato da Airbus per i propri calcoli del range.

In questo modo è stato possibile ricavare la massa del pieno carico di carburante, pari a 19625 kg, e la frazione di carburante con massa massima al decollo, pari a 0.260. Si è quindi potuto calcolare il rapporto della massa a fine missione semplicemente invertendo l'equazione \ref{eq:frazione_carburante}:
\[\frac{m_{final}}{m_{TO}} = 1-\dfrac{1}{1.05}\dfrac{m_{fuel}}{m_{TO}}\]

Si è quindi calcolato il range della crociera a partire dall'equazione dell'autonomia chilometrica (\ref{eq:autonomia}), considerando che la frazione di massa finale è pari al prodotto delle frazioni di massa in ogni fase della missione (come da equazione \ref{eq:w_final}) e che tutte le altre frazioni restano uguali al punto armonico:
\[
\text{Range}_C = -\ln\left(\dfrac{\dfrac{m_{final}}{m_{TO}}}{\dfrac{W_2}{W_{TO}} \cdot \dfrac{W_3}{W_2} \cdot \dfrac{W_5}{W_4} \cdot \dfrac{W_8}{W_7} \cdot \dfrac{W_{10}}{W_9}}\right) \cdot \dfrac{v_{cruise} \cdot E_{cruise}}{SFC_{cruise}} = 6575\,km
\]

\noindent Il payload corrisponde invece a:
\[
m_{payload_C} = m_{TO} - (m_{empty} + m_{fuel_{max}} + m_{crew}) = 12739 kg
\]
pari a 137 passeggeri.

\medskip


\noindent Infine, per il calcolo del range dell'ultimo punto si è calcolata la massa al decollo, pari a
\[
m_{TO_D} = m_{empty} + m_{fuel_{max}} + m_{crew} = 57102 kg
\]
dove è stata considerata una crew di 2, ovvero solo i piloti, in quanto non essendoci passeggeri non c'è necessità di personale di bordo.

Successivamente il calcolo è uguale a quello per il punto C: si è calcolata la frazione di carburante (pari a 0.344) ed il range della crociera alla stessa maniera: il valore che si ricava è di $9356\,km$, con payload ovviamente pari a 0.

\begin{figure}[H]
    \centering
    \includegraphics[width=0.76\linewidth]{E1//Immagini/payload-range.png}
    \caption{Diagramma payload-range dell'aeroplano.}
\end{figure}

\subsubsection{Famiglia di aerei}
Si è ipotizzato di voler modificare l'aeroplano in modo da creare una famiglia, sulla falsa riga di quella dell'Airbus A320 con A319 e A321, quindi semplicemente modificando la fusoliera in modo da accogliere diversi numeri di passeggeri. In particolare, si è ipotizzata una configurazione a 150 passeggeri ed un'altra a 210 passeggeri. È stata inoltre considerata una configurazione a 210 passeggeri e range esteso, ipotizzando che grazie alla fusoliera più grande ci sia spazio per dei serbatoi aggiuntivi, in particolare fino a 35000 litri di carburante. 

In maniera del tutto analoga per il caso a 180 passeggeri - con l'unica differenza nel numero di assistenti di bordo in quanto da normativa ne bastano 3 per il primo caso e ne servono 5 per il secondo - si sono calcolate le masse ed i range per l'intera famiglia.

\begin{figure}[H]
    \centering
    \includegraphics[width=0.95\linewidth]{E1//Immagini/payload-range_family.png}
    \caption{Diagramma payload-range della famiglia di aerei.}
\end{figure}

Si notano ovvie differenze di range, dovute innanzitutto alla differenza di payload, ma anche alla differenza di peso a vuoto.

\begin{table}[H]
    \centering
    \begin{tabular}{c|c|c|c|c|c}
    \toprule
         Passeggeri & $range_C$ [km] & $range_D$ [km] & $m_{TO}$ [kg] & $m_{empty}$ [kg] & $m_{empty}/m_{TO}$ \\
         \midrule
         150 & 8524 & 10825 & 55754 & 28072 & 0.504 \\
         180 & 6575 & 9356 & 65657 & 32735 & 0.499 \\
         210 & 5239 & 8210 & 75418 & 37291 & 0.495 \\
         210 (ER) & 8997 & 11377 & 75418 & 37291 & 0.495 \\
         \bottomrule
    \end{tabular}
    \caption{Valori di range e masse per la famiglia di aerei.}
\end{table}

\vspace{1cm}
\subsection{Matching Chart}
A partire dai risultati preliminari è stato creato il \textit{matching chart}. Si tratta di un grafico in cui sono riportate le curve della spinta necessaria (rapportata al peso), in diverse fasi di volo, in funzione del carico alare. Se ne ricava la spinta minima da installare sul veivolo in modo da soddisfare la normativa.

\subsubsection{Carico alare massimo}
Il primo parametro da determinare è il carico alare massimo, che si ricava dalla formula della portanza alla velocità minima, ovvero quella di stallo:
\[
L = W = \dfrac{1}{2}\rho V^2 S C_L \implies \left(\dfrac{W}{S}\right)_{max} = \dfrac{1}{2}\rho V_{stall}^2 C_{L_{max}}
\]
La velocità di stallo è imposta a 115 nodi, in linea con gli aerei della stessa categoria, ed aumentata di un fattore 1.2 per simulare le condizioni di volo a bassa velocità.

Per il coefficiente di portanza massimo è stato necessario scegliere un profilo alare: il \textbf{NASA SC(2)-0610}, ovvero un profilo supercritico adatto al volo transonico. Tramite il tool online \textit{airfoiltools.com} ne è stato ricavato il coefficiente di portanza massimo, pari a 1.7. 
\begin{figure}[H]
    \centering
    \includegraphics[width=0.95\linewidth]{E1//Immagini/profilo alare.png}
    \caption{Profilo NASA SC(2)-0610}
\end{figure}
Dal coefficiente di portanza del profilo si è passati a quello dell'ala:
\[C_{L_{max}}^{3D} = 0.9\,C_{L_{max}}^{2D}\cos{\Lambda_{25}}\]
dove $\Lambda_{25}$ è l'angolo di freccia al quarto di corda, in questa fase preliminare fissato a 25 gradi.

Questo valore del $C_L$ va aumentato del contributo dato dai dispositivi di ipersostentazione, in questo caso slat al bordo d'attacco e Fowler flap al bordo di fuga che producono un incremento $\Delta C_L^{2D} = 1.55$. Per calcolare il contributo dato all'ala si usa la seguente formula:
\[\Delta C_L^{3D} = 0.92\,\Delta C_L^{2D}\,\dfrac{S_{flapped}}{S}\cos{\Lambda_{25}}\]
Il valore $S_{flapped}$ rappresenta l'intera superficie alare su cui sono installati gli ipersostentatori; il rapporto con la superficie alare è stato calcolato pari a 0.85 da misure dell'ala dell'Airbus A320.
\newline
\noindent Il coefficiente di portanza massimo così calcolato è pari a 2.49.

In definitiva, si ricava un carico alare massimo di $715.5 \,kg/m^2$.
\subsubsection{Decollo}
Per calcolare la spinta richiesta al decollo si è usato l'approccio semplificato elaborato da Raymer:
\[\left(\dfrac{T}{W}\right)_{take-off} = \dfrac{W/S}{TOP\,\sigma\,C_L^{take-off}}\]
dove: $\sigma$ è il rapporto tra la densità dell'aria alla quota dell'aeroporto e a livello del mare, in questo caso imposto a 1; come $C_L^{take-off}$ si è usato il coefficiente di portanza massimo, già ricavato per il calcolo del carico alare massimo; \textit{TOP} è il \textit{Take Off Parameter}, un parametro calcolato semi-empiricamente da Raymer e ricavato dal seguente grafico:
\begin{figure}[H]
    \centering
    \includegraphics[width=0.65\linewidth]{E1//Immagini/Take Off Parameter.png}
    \caption{Grafico del Take Off Parameter}
\end{figure}
Si entra nel grafico con la distanza di decollo, per cui è stata usata la \textit{Balanced Field Length} dell'A320 pari a 6900 ft e, intercettando la corretta curva relativa a due motori a getto, si legge il TOP pari a $165 \,\,lb/ft^2$, convertito nel sistema internazionale per effettuare i calcoli.

\subsubsection{Salita}
L'equazione per il rapporto di spinta è ricavata dalla meccanica del volo:
\begin{equation}
    \dfrac{T}{W} = k_{OEI}\left(\frac{1}{2}\rho_{SL}\dfrac{V^2}{W/S}\left(C_D+\text{K}\left(\dfrac{W/S}{\rho_{SL}V^2}\cos{\gamma}\right)^2\right)+\sin{\gamma}\right)
    \label{eq:spinta_salita}
\end{equation}
dove: $ k_{OEI}$ è il parametro di \textit{One Engine Inoperative}, nel caso di aereo bimotore pari a 2; la densità dell'aria è approssimata a quella a livello del mare, anche se aumentando di quota diminuisce; K$\, = \frac{1}{\pi\,AR\,e}$ è il parametro per il calcolo della resistenza indotta $C_{D_i} = \text{K}C_L^2$, con il fattore di Oswald imposto pari a 0.8 in questa fase preliminare. \newline Il $C_{D_0}$ è stato imposto pari a 0.017 a partire da dati statistici per velivoli della stessa categoria.

\vspace{0.5cm}
Da normativa la fase di salita è divisa in tre segmenti, e il veivolo deve soddisfare i seguenti requisiti con un motore inoperativo:
\begin{table}[H]
    \centering
    \renewcommand{\arraystretch}{1.2}
    \begin{tabular}{c|c|c|c}
       & Primo segmento & Secondo segmento & Terzo segmento \\
    \toprule
        Comincia & fine pista & carrello retratto & configurazione finale \\
    \midrule
        Gradiente minimo $\gamma$ & 0.0\% & 2.4\% & 1.2\% \\
    \midrule
       Configurazione ipersostentatori  & decollo & decollo & retratti \\
    \midrule
       Carrello  & in retrazione & retratto & retratto \\
    \midrule
       Velocità & $V_2$ & $V_2$ & $1.25\,V_2$ \\
    \end{tabular}
\end{table}
\noindent La velocità $V_2$ è quella caratteristica del decollo, in questa fase posta pari a 1.2 volte la velocità di stallo.

\noindent Gli ipersostentatori aumentano il coefficiente di resistenza:
\[C_{D,flap} = 0.9\left(\dfrac{c_{flap}}{c}\right)^{1.38}\left(\dfrac{S_{w,flap}}{S}\right)\sin^2{\delta_{flap}}\]
dove i rapporti tra le corde e le superfici sono stati ricavati dai dati dell'ala dell'A320 e $\delta_{flap}$ rappresenta l'angolo di deflessione dei flap, in fase di decollo pari a $35\si{\degree}$.

Anche il carrello d'atterraggio aumenta il coefficiente di resistenza:
\[C_{D,LG} = \num{2.92e-3}\dfrac{W_{TO}^{0.785}}{S}\]
dove il valore del peso al decollo è quello ricavato dal calcolo preliminare, e il valore della superficie alare è ricavato da quello del carico alare.
\vspace{0.5cm}
Per rappresentare sul matching chart la spinta richiesta per la salita, è stata considerata la condizione più gravosa tra i tre segmenti, che corrisponde al secondo segmento in quanto è richiesto un gradiente maggiore.

\subsubsection{Crociera}
Anche in questo caso l'equazione per la spinta richiesta è ricavata dalla meccanica del volo:
\[\dfrac{T}{W} = \dfrac{\dfrac{1}{2}\rho_{cruise}V^2}{W/S}\left(C_{D_0}+\text{K}\left(\dfrac{W/S}{\rho_{cruise}V^2}\right)^2\right)\]
dove la velocità e la quota di crociera sono, in questa fase, imposte da requisito rispettivamente a 835 km/h e a 33000 piedi, che con il modello dell'aria standard corrisponde ad una densità di 0.4097 kg/m$^3$.

Si è infine riferita la spinta richiesta al livello del mare per poterla paragonare con le altre fasi di volo:
\[\left(\dfrac{T}{W}\right)_{SL} = \dfrac{\left(\dfrac{T}{W}\right)_{cruise}}{\dfrac{\rho_{cruise}}{\rho_{SL}}}\]

\subsubsection{Approach Climb}
È la condizione per cui si verifica un guasto ad un motore durante la fase di approccio (carrello retratto e flap a $20\si{\degree}$), ad una velocità pari a 1.41 volte la velocità di stallo. Il gradiente di salita richiesto è del 2.1\%. \newline L'equazione per ricavare il rapporto di spinta è esattamente la stessa per la fase di salita (equazione \ref{eq:spinta_salita}). 

\subsubsection{Landing Climb}
È la condizione per cui si verifica un guasto ad un motore durante la fase di atterraggio (carrello esteso e flap a $35\si{\degree}$), ad una velocità pari a 1.23 volte la velocità di stallo. Il gradiente di salita richiesto è del 3.2\%. \newline L'equazione per ricavare il rapporto di spinta è esattamente la stessa per la fase di salita (equazione \ref{eq:spinta_salita}). 

\subsubsection{Matching Chart}
\begin{figure}[H]
    \centering
    \includegraphics[width=\linewidth]{E1//Immagini/Matching Chart.png}
    \caption{Matching Chart dell'aeroplano.}
\end{figure}
Ricavata la spinta minima richiesta per ogni fase di volo, si può definire la spinta minima da installare sul velivolo, come la massima tra le spinte richieste in ogni fase. In generale è preferibile installare la spinta minore possibile per ridurre il peso del sistema propulsivo ed il consumo di carburante.

\vspace{1cm}
\subsection{Fusoliera}
La prima parte del velivolo da disegnare, in quanto uguale per tutte le diverse configurazioni aerodinamiche e propulsive, è la fusoliera. La sua funzione primaria è accogliere il payload (passeggeri, bagagli e cargo); da requisito deve poter ospitare 180 passeggeri e relativi bagagli, 4 assistenti di volo e 2 piloti. 
\subsubsection{Diametro esterno}
Si è scelto, in questa fase di design preliminare, di mantenere una sezione circolare per tutta la lunghezza della fusoliera, nonostante nella realtà ciò è vero solo per i velivoli \textit{wide body}. \newline Si è ipotizzata una larghezza del singolo sedile pari a 0.54 metri, standard per sedili di classe economica. Con sei sedili per fila e la larghezza del corridoio pari a 0.50 metri, nel range di valori soliti per i velivoli \textit{narrow body}, si ricava il diametro interno pari a 3.74 metri; si posizionano i sedili in modo tale che la loro larghezza massima corrisponda al diametro massimo della sezione. \newline Si è inoltre considerato uno spessore ulteriore per la struttura stessa della fusoliera e per la pannellatura interna, che porta il diametro esterno della fusoliera a \textbf{4.10 metri}.
\subsubsection{Lunghezza}
Il numero di passeggeri permette di stimare la lunghezza della cabina, ad esclusione della sezione di coda e di prua: si è assunto un \textit{pitch} tra i sedili pari a 0.72 metri, e calcolato il numero di file semplicemente dividendo il numero di passeggeri per il numero di sedili a fila, ovvero sei, ricavando così 30 file e quindi una lunghezza di 21.60 metri. 

In realtà facendo così si sovrastima leggermente la lunghezza della cabina, in quanto alcune file rientrano nella parte di coda e di prua, ma ciò permette di considerare anche lo spazio in più necessario in cabina per elementi come i sedili degli assistenti di volo e le toilette.\newline La lunghezza delle sezioni di poppa e di prua è stata stimata grazie a formule statistiche: la parte di poppa pari a 2.5 volte il diametro esterno, quella di prua pari a 1.5 volte il diametro esterno: le loro lunghezze sono rispettivamente 6.15 e 10.25 metri. \newline La lunghezza totale della fusoliera è quindi di \textbf{38 metri}.

\subsubsection{Stiva}
Per calcolare il volume della stiva si è divisa la fusoliera lasciando il giusto spazio in cabina per il comfort dei passeggeri, ed inoltre si è omesso lo spazio per ospitare il carrello d'atterraggio, la \textit{wingbox} in corrispondenza dell'ala e altri sistemi. Si è così ottenuto un volume di $\mathbf{43 \,\,\textbf{m}^3}$. 

Successivamente, si è stimato il volume dei bagagli dei passeggeri, in base a dati statistici, pari a 0.113 $m^3$ per bagaglio, ottenendo quindi 20.34 $m^3$ del volume della stiva occupato dai bagagli dei passeggeri.

Il rimanente spazio è stato riservato per le merci: in particolare per container \textbf{LD3/46(W)}, standard per questa categoria di aerei, il cui volume è di 4.04 $m^3$. Si sono quindi potuti inserire 5 container, per un totale di 20.20 $m^3$.

\vspace{1cm}
\subsection{Analisi di sensibilità}
A partire dai risultati preliminari, è stata impostata un'analisi di sensibilità in modo da ricavare diverse configurazioni. In particolare, sono stati scelti alcuni parametri aerodinamici come variabili di design, e fatti variare entro un range ragionevole:

\begin{table}[H]
    \centering
    \renewcommand{\arraystretch}{1.2}
    \begin{tabular}{|c|c|c|}
    \hline
        \rowcolor{grey}
        Variabile  & Simbolo & Valori \\
    \hline
        Carico Alare & \textit{W/S} & $[550\, 600\, 650\, 700]\,\, kg/m^2$ \\
    \hline
        Aspect Ratio & AR & $[7\, 8\, 9\, 10\, 11]$ \\
    \hline
       Rapporto Spessore Alare - Corda  & \textit{t/c} & $[0.10\, 0.12\, 0.15]$ \\
    \hline
       Mach di Crociera  & M & $[0.76\, 0.80\, 0.82]$ \\
    \hline
       Angolo di Freccia al quarto di corda & $\Lambda_{25}$ & $[20\, 25\, 30\, 35]\,\,\si{\degree} $ \\
    \hline
       Rapporto di Rastremazione & $\lambda$ & $[0.23\, 0.27\, 0.31]$ \\
    \hline
    \end{tabular}
    \caption{Variabili di design}
\end{table}

Sono stati quindi calcolati i parametri di peso, aerodinamici e di prestazione per ogni combinazione delle variabili di design. Per fare ciò si è impostato su MATLAB un ciclo di convergenza sul peso massimo al decollo, in quanto è un parametro che influenza e viene influenzato da tutti gli altri. Si è imposta una tolleranza sulla differenza relativa di 25 kg e si è partiti dal valore iniziale calcolato in via preliminare, ovvero 65657 kg.
\vspace{0.5cm}
\newline Il primo step in ogni fase del ciclo è quello di calcolare mediante il matching chart il valore della spinta da installare per quella configurazione, funzione del carico alare. I passaggi successivi sono il calcolo dei parametri aerodinamici, la stima dei pesi, delle prestazioni in missione ed infine, una volta giunto a convergenza il ciclo, i costi associati alla particolare configurazione.

\subsubsection{Aerodinamica}
L'obiettivo di questa sezione del calcolo è quello di trovare il coefficiente di resistenza del velivolo. I tre principali contributi alla resistenza sono la resistenza a portanza nulla $C_{D_0}$, a cui contribuisce l'intero velivolo, la resistenza indotta $C_{D_i}$ e la resistenza d'onda $C_{D_W}$, generate dal funzionamento dell'ala:\[C_D=C_{D_0}+ C_{D_i} + C_{D_W}\]

\paragraph{\texorpdfstring{$\mathbf{C_{D_0}}$}{CD0}}
    
    È calcolato come somma dei contributi del $C_{D_0}$ di ala, fusoliera, impennaggi di coda e gondole motore. È stato calcolato secondo la formula derivata da Raymer:
    \[C_{D_0} = \dfrac{\sum_i c_{F_i}\cdot FF_i\cdot Q_i\cdot S_{W_i}}{S}\]
    Dove: $c_F$ è il coefficiente di attrito equivalente, $FF$ il fattore di forma, $Q$ il fattore di interferenza e $S_W$ la superficie bagnata, mentre $S$ è la superficie alare. Essa è calcolata ad ogni iterazione del ciclo come rapporto tra il peso al decollo e il carico alare; l'apertura alare $b$ è calcolata come radice quadrata del prodotto tra $Aspect Ratio$ e superficie alare.

    Il coefficiente d'attrito equivalente per un flusso turbolento è pari a:
    \[ c_{F_{TURB}} = \frac{0.455}{\left[ \log_{10}(Re) \right]^{2.58} \left[ 1 + 0.144 \cdot M^2 \right]^{0.65}} \]
    dove $M$ è il Mach di crociera, variabile di design, e $Re$ il numero di Reynolds calcolato nelle condizioni di crociera: velocità pari al Mach di crociera per velocità del suono alla quota di crociera (pari a $316.01\,m/s$) e viscosità dinamica calcolata tramite le relazioni dell'$International\,\,Standard\,\,Air$, pari a \num{1.591e-5} $kg/(m\cdot s)$; la lunghezza caratteristica $L$ è differente per ogni componente.
    
    A seguire i parametri per il calcolo del $C_{D_0}$ di ogni componente:
        \begin{itemize}
            \item \textbf{Ala}:
            \begin{table}[H]
                \centering
                \renewcommand{\arraystretch}{1.2}
                \begin{tabular}{c|c}
                    L & $MAC = \dfrac{2}{3}c_{root}\dfrac{1+\lambda+\lambda^2}{1+\lambda}$  ;  $c_{root} = \dfrac{S}{\dfrac{b-d_{fus}}{2}\left(1+\lambda\right)}$ \\
                    \midrule \\
                    FF & $\left[ 1 + \dfrac{0.6}{(x/c)_m} \cdot \left( \dfrac{t}{c} \right) + 100 \cdot \left( \dfrac{t}{c} \right)^4 \right] \cdot \left[ 1.34 \cdot M_{cruise}^{0.18} \cdot (\cos \Lambda_{25})^{0.28} \right]$ ; $(x/c)_m \approx 0.5$ \\
                    \midrule \\
                    Q & 1.0 \\
                    \midrule \\
                    $S_W$ & $S\left[1.997+0.52\left(\dfrac{t}{c}\right)\right]$
                \end{tabular}
             \end{table}
            \item \textbf{Fusoliera}:
            \begin{table}[H]
                \centering
                \renewcommand{\arraystretch}{1.2}
                \begin{tabular}{c|c}
                    L & $l_{fus} = 38.0 \,m$ \\
                    \midrule \\
                    FF & $\left(1+\dfrac{60}{\left(l_{fus}/d_{fus}\right)^3}+\dfrac{\left(l_{fus}/d_{fus}\right)}{400}\right)$ \\
                    \midrule \\
                    Q & 1.0 \\
                    \midrule \\
                    $S_W$ & $\pi\left(\dfrac{d_{fus}^2}{2}+d_{fus}\left(l_{fus}-\dfrac{d_{fus}}{2}-l_T\right)\right)+\left(d_{fus}+\pi\right)*\dfrac{l_t}{2}$
                \end{tabular}
            \end{table}
            dove la superficie della fusoliera è stata divisa in: una semisfera in prua, un cilindro centrale e una superficie trapezoidale per la sezione di coda.

            \item \textbf{Equilibratore}:
            \begin{table}[H]
                \centering
                \renewcommand{\arraystretch}{1.2}
                \begin{tabular}{c|c}
                    L & $b_{orizz} = \sqrt{AR_{orizz}S_{orizz}}$ \\
                    \midrule \\
                    FF & $\left[ 1 + \dfrac{0.6}{(x/c)_m} \cdot \left( \dfrac{t}{c} \right)_{orizz} + 100 \cdot \left( \dfrac{t}{c} \right)_{orizz}^4 \right] \cdot \left[ 1.34 \cdot M^{0.18}_{cruise} \cdot (\cos \Lambda_{25_{orizz}})^{0.28} \right]$ ; $(x/c)_m \approx 0.5$ \\
                    \midrule \\
                    Q & 1.05 \\
                    \midrule \\
                    $S_W$ & $S\left[1.997+0.52\left(\dfrac{t}{c}\right)_{orizz}\right]$
                \end{tabular}
             \end{table}
             Per i parametri geometrici dell'equilibratore si è fatto riferimento a dati statistici per velivoli della stessa categoria:
             \begin{table}[H]
                \centering
                \renewcommand{\arraystretch}{1.2}
                \begin{tabular}{c|c}
                    $S_{orizz}$ & $0.25\, S_{wing}$\\
                    $AR_{orizz}$ & 5.0 \\
                    $\Lambda_{25_{orizz}}$ & $29\si{\degree}$ \\
                    $\left(\dfrac{t}{c}\right)_{orizz}$ & 0.10
                \end{tabular}
            \end{table} 

            \item \textbf{Timone}:
            \begin{table}[H]
                \centering
                \renewcommand{\arraystretch}{1.2}
                \begin{tabular}{c|c}
                    L & $b_{vert} = \sqrt{AR_{vert}S_{vert}}$ \\
                    \midrule \\
                    FF & $\left[ 1 + \dfrac{0.6}{(x/c)_m} \cdot \left( \dfrac{t}{c} \right)_{vert} + 100 \cdot \left( \dfrac{t}{c} \right)_{vert}^4 \right] \cdot \left[ 1.34 \cdot M^{0.18}_{cruise} \cdot (\cos \Lambda_{25_{vert}})^{0.28} \right]$ ; $(x/c)_m \approx 0.5$ \\
                    \midrule \\
                    Q & 1.05 \\
                    \midrule \\
                    $S_W$ & $S\left[1.997+0.52\left(\dfrac{t}{c}\right)_{vert}\right]$
                \end{tabular}
             \end{table}
             Anche i parametri geometrici del timone sono stati ricavati da dati statistici per velivoli della stessa categoria:
             \begin{table}[H]
                \centering
                \renewcommand{\arraystretch}{1.2}
                \begin{tabular}{c|c}
                    $S_{vert}$ & $0.18 \,S_{wing}$\\
                    $AR_{vert}$ & 1.8 \\
                    $\Lambda_{25_{vert}}$ & $34\si{\degree}$ \\
                    $\left(\dfrac{t}{c}\right)_{vert}$ & 0.12
                \end{tabular}
            \end{table} 

            \item \textbf{Gondole motore}:
            \begin{table}[H]
                \centering
                \renewcommand{\arraystretch}{1.2}
                \begin{tabular}{c|c}
                    L & $l_{nac} = 0.07 \sqrt{T/2}$  \\
                    \midrule \\
                    FF & $1+\dfrac{60}{f_{nac}^3}+\dfrac{f_{nac}}{400}$ ; $f_{nac} = \dfrac{l_{nac}}{\sqrt{(4/\pi)A_{nac}}}$ \\
                    \midrule \\
                    Q & 1.0 \\
                    \midrule \\
                    $S_W$ & $\pi d_{nac}l_{nac}$ ; $d_{nac} = 0.04 \sqrt{T/2}$
                \end{tabular}
            \end{table}
            dove la lunghezza ed il diametro della nacella è stata ricavata da una formula empirica, in cui la spinta è espressa in libbre ed il risultato è in piedi, opportunamente convertiti. Il fattore di forma è stato considerato simile a quello della fusoliera.
        \end{itemize}
        \vspace{0.6cm}

        \paragraph{\texorpdfstring{$\mathbf{C_{D_i}}$}{CDi}}
    
    È calcolato come: \[C_{D_i} = K C_L^2\]
    dove K è il parametro adimensionale: \[K = \dfrac{1}{\pi AR\, e}\]
    A sua volta, il Fattore di Oswald $e$ è ricavato da una formula statistica interpolativa, dipendente dal rapporto di rastremazione: \[e = \dfrac{\cos(\Lambda_{25})}{1 + AR \cdot f_\lambda}\] 
    dove $f_\lambda$ è calcolato come: \[f_\lambda = 0.0524 \lambda^4 - 0.15 \lambda^3 + 0.1659 \lambda^2 - 0.0706 \lambda + 0.0119\]
    Infine, il coefficiente di portanza è calcolato direttamente dall'equazione della portanza in condizioni di crociera: \[C_L = \dfrac{2\,\,W/S}{\rho_{cruise} V_{cruise}^2}\]

    \paragraph{\texorpdfstring{$\mathbf{C_{D_W}}$}{CDW}}
    La resistenza d'onda è direttamente legata all'angolo di freccia, che riduce il Mach del flusso incidente l'ala ed aumenta il Mach critico, a cui si iniziano a verificare fenomeni supersonici sul profilo. Esso è calcolato come: \begin{align*} 
        M_{dd} = \dfrac{0.90}{\cos \Lambda_{25}} &- \dfrac{(t/c)}{\cos^2 \Lambda_{25}} - \dfrac{0.10 \,C_L}{\cos^3 \Lambda_{25}} \\ \\
        M_{crit} = M_{dd} &- \left( \dfrac{0.1}{80} \right)^{1/3} 
    \end{align*}

    Se il Mach di crociera è minore del Mach critico, la resistenza d'onda è pari a zero; altrimenti, essa è calcolata come: \[C_{D_W} = 20 \cdot \left(M_{cruise} - M_{crit}\right)^4\]

Una volta ricavato il coefficiente di resistenza, è possibile calcolare l'efficienza aerodinamica in crociera, un parametro importante per la valutazione delle prestazioni del velivolo: \[E = \dfrac{L}{W} = \dfrac{C_L}{C_D}\]

\vspace{0.6cm}

\subsubsection{Stima dei Pesi}

La stima dei pesi rappresenta una parte fondamentale nella progettazione preliminare del velivolo, in quanto consente di definire i valori di massa richiesti per soddisfare i requisiti operativi e strutturali. In questa sezione, si dettagliano i principali componenti della massa operativa a vuoto (OEW, \textit{Operating Empty Weight}) e le relative dipendenze dai parametri di progetto. Tutte le formule sono ricavate empiricamente, sulla base di dati statistici di velivoli già esistenti, da Raymer; di conseguenza, tutti i parametri sono espressi, nelle equazioni, nel sistema imperiale.
\paragraph{Ala}

Il peso dell'ala dipende da diversi parametri geometrici e strutturali, sintetizzati nell'indice geometrico dell'ala ($I_w$):

\begin{equation*}
I_w = \frac{n_{ult} \cdot AR^{1.5} \cdot \sqrt{\dfrac{W_{\text{zf}}}{W_{\text{TO}}}} \cdot (1 + 2\lambda) \cdot W/S \cdot S^{1.5} \cdot 10^{-6}}{\left(t/c\right) \cdot \cos^2 \Lambda_{25} \cdot (1 + \lambda)}
\end{equation*}

Dove:
\begin{itemize}
    \item $n_{\text{ult}}$: \textit{ultimate load factor}, fissato a 3.75;
    \item $W_{\text{zf}}$: peso a zero carburante $W_{\text{zf}} = W_{\text{TO}} \cdot (1 - f_{\text{fuel}})$, con la frazione di carburante $f_{\text{fuel}} = \dfrac{W_{\text{fuel}}}{W_{\text{TO}}}$ calcolata inizialmente nel calcolo preliminare del peso del velivolo, poi ricalcolata ad ogni step del ciclo di convergenza;
    \item $W_{\text{TO}}$: peso massimo al decollo.
\end{itemize}

In base al valore di $S$, il peso dell'ala viene calcolato come:

\begin{equation*}
W_{\text{wing}} = \begin{cases} 
    (0.93 \cdot I_w + 6.44 \cdot S + 390) & \text{se } S \geq 900 \ \text{ft}^2, \\
    (4.24 \cdot I_w + 0.57 \cdot S) & \text{altrimenti}.
\end{cases}
\end{equation*}

\paragraph{Coda}

Il peso della coda convenzionale, incluse sia le superfici orizzontali e che quella verticale, è calcolato come:

\begin{equation*}
W_{\text{tail}} = 5.03 \cdot (S_{\text{orizz}} + S_{\text{vert}})
\end{equation*}

\paragraph{Fusoliera}

Il peso della fusoliera dipende dalle sue dimensioni:

\begin{equation*}
W_{\text{fus}} = 1.35 \cdot \left( L_{\text{fus}} \cdot D_{\text{fus}} \right)^{1.28}
\end{equation*}

Dove:
\begin{itemize}
    \item $L_{\text{fus}}$: lunghezza della fusoliera;
    \item $D_{\text{fus}}$: diametro esterno della fusoliera.
\end{itemize}

\paragraph{Carrello di atterraggio}

Il peso del carrello di atterraggio è calcolato considerando quattro contributi principali, tutti funzione del peso massimo al decollo ($W_{\text{TO}}$):

\begin{equation*}
W_{\text{LG}} = W_{\text{LG, strutt}} + W_{\text{LG, freni}} + W_{\text{LG, pneumatici}} + W_{\text{LG, controllo}}
\end{equation*}

I contributi individuali sono determinati come segue:
\begin{align*}
        W_{\text{LG, strutt}} &= W_{\text{LG, param}} \cdot \left( 0.45 + 23.1 \cdot 10^{-8} \cdot W_{\text{TO}} \right) \\
        W_{\text{LG, freni}} &= W_{\text{LG, param}} \cdot \left( 0.268 - 8.12 \cdot 10^{-8} \cdot W_{\text{TO}} \right) \\
        W_{\text{LG, pneumatici}} &= W_{\text{LG, param}} \cdot \left( 0.152 - 8.38 \cdot 10^{-8} \cdot W_{\text{TO}} \right) \\
        W_{\text{LG, controllo}} &= W_{\text{LG, param}} \cdot \left( 0.13 - 6.56 \cdot 10^{-8} \cdot W_{\text{TO}} \right) \\
\end{align*}

Il parametro $W_{\text{LG, param}}$ è definito come:

\begin{equation*}
W_{\text{LG, param}} = 0.044 \cdot W_{\text{TO}} - 672
\end{equation*}

\paragraph{Propulsione}

Il peso del sistema propulsivo comprende motori e gondole:

\begin{equation*}
W_{\text{nac}} = \frac{1}{4} \cdot N_{\text{prop}} \cdot D_{\text{nac}} \cdot L_{\text{nac}} \cdot \left( T_{\text{engine}} \right)^{0.36}
\end{equation*}
dove $N_{\text{prop}}$ è il numero di motori, pari a due, e $T_{\text{engine}}$ la spinta del singolo motore, quindi quella totale installata divisa per il numero di motori.
\begin{equation*}
W_{\text{engine}} = \frac{\left(T/W\right) \cdot W_{\text{TO}}}{5.5}
\end{equation*}

\begin{equation*}
W_{\text{prop}} = W_{\text{nac}} + W_{\text{engine}}
\end{equation*}

\paragraph{Sistemi}

Il peso dei sistemi include i seguenti contributi:

\begin{itemize}
    \item \textbf{Sistema carburante:} \begin{equation*} W_{\text{fuelsys}} = 2.71 \cdot \left( \frac{b}{\cos \Lambda_{25}} \cdot N_{\text{serbatoi}} \right)^{0.956} \end{equation*}
    con il numero di serbatoi imposto pari a sei, ovvero due in ogni semiala, uno centrale in fusoliera ed un \textit{trim tank} nell'equilibratore.
    \item \textbf{Sistema idraulico:} \begin{equation*} W_{\text{hydraulic}} = \begin{cases} 
        45 + 1.318 \cdot S_{\text{ref}} & \text{se } S_{\text{ref}} \leq 3000 \ \text{ft}^2, \\
        18.7 \cdot S_{\text{ref}}^{0.712} - 1620 & \text{altrimenti}.
    \end{cases} \end{equation*}
    con 
    \begin{equation*}
        S_{\text{ref}} = S_{\text{wing}} + 1.44\cdot(S_{\text{orizz}}+S_{\text{vert}})
    \end{equation*}
    \item \textbf{Sistema elettrico:} \begin{equation*} W_{\text{elec}} = 16.2 \cdot N_{\text{passeggeri}} + 110 \end{equation*}
    \item \textbf{Sistema pneumatico:} \begin{equation*} W_{\text{pneumatic}} = 26.2 \cdot N_{\text{passeggeri}}^{0.944} \end{equation*}
    \item \textbf{Sistema anti-ghiaccio}, senza anti-ghiaccio sulle superfici di coda: \begin{equation*} W_{\text{anti-ice}} = 0.120 \cdot S_{\text{wing}} \end{equation*}
    \item \textbf{Strumentazione} per la manetta, per il carburante, e altra: \begin{equation*} W_{\text{instr}} = \left( 0.00145 \cdot T_{\text{engine}} + 30 \right) \cdot N_{\text{prop}} + \left( 0.00714 \cdot V_{\text{fuel}} + 34 \right) + (1.872 \cdot N_{\text{passeggeri}} + 128) \end{equation*}
    \item \textbf{Avionica:} \begin{equation*} W_{\text{avionics}} = 2.8 \cdot N_{\text{pax}} + 2320 \end{equation*}
    \item \textbf{Sistema motore:} \begin{equation*} W_{\text{engine sys}} = 133 \cdot N_{\text{prop}} \end{equation*}
\end{itemize}

\paragraph{Arredi e Servizi}

Il peso degli arredi e dei servizi per i passeggeri è calcolato come:

\begin{equation*}
W_{\text{furn}} = 118.4 \cdot N_{\text{passeggeri}} - 4190
\end{equation*}

Il peso dei servizi include contributi come catering e assistenza ai passeggeri:

\begin{equation*}
W_{\text{services}} = 2.529 \cdot N_{\text{passeggeri}} \cdot \left(\frac{\text{Range}}{M_{\text{cruise}}}\right)^{0.225}
\end{equation*}

\paragraph{Equipaggio}

Il peso dell'equipaggio è calcolato considerando i due piloti ed i quattro assistenti di volo:

\begin{equation*}
W_{\text{crew}} = 2 \cdot 225 + 4 \cdot 155
\end{equation*}

\paragraph{OEW totale}

Il peso operativo a vuoto è infine calcolato sommando tutti i contributi:

\begin{equation*} 
    \begin{split} 
        \text{OEW} = &\, W_{\text{wing}} + W_{\text{tail}} + W_{\text{fus}} + W_{\text{LG}} + W_{\text{prop}} + W_{\text{fuelsys}} + \\
         &\, W_{\text{hydraulic}} + W_{\text{elec}} + W_{\text{pneumatic}} + W_{\text{anti-ice}} + W_{\text{instr}} + \\
         &\, W_{\text{avionics}} + W_{\text{engine sys}} + W_{\text{furn}} + W_{\text{services}} + W_{\text{crew}}
    \end{split} 
\end{equation*}

\paragraph{Payload}
Il carico pagante del velivolo è composto da passeggeri e relativi bagagli, e merci. 

Come anticipato nel calcolo del peso preliminare, il peso di un singolo passeggero è 79.4 kg e il suo bagaglio, per un volo di corto-medio raggio, 13.6 kg; ciò porta ad una massa totale dei passeggeri di $\mathbf{16740 \textbf{ kg}}$.

Calcolando il volume della stiva dell'aereo, si è stimato che possa contenere cinque container \textbf{LD3/46(W)}, standard per questa categoria di aerei, il cui peso lordo massimo è di 1135 kg, per un totale del peso delle merci di $\mathbf{5675 \textbf{ kg}}$. 

Il peso totale del payload si attesta quindi a $\mathbf{22415 \textbf{ kg}}$.

\subsubsection{Prestazioni in Missione}
L'ultimo step del ciclo è ricalcolare le prestazioni in missione, in particolare il consumo di carburante. Per ciò, si riutilizzano le formule di Breguet già usate nel calcolo preliminare, ma stavolta con la "vera" efficienza aerodinamica. si ricalcolano in particolare le differenze di peso (quindi i consumi di carburante) in crociera, loiter e diversione. 
\begin{align*}
    \text{Autonomia chilometrica: }&\dfrac{W_{f}}{W_i} = e^{-\dfrac{c\Delta x}{VE}}\\
    \text{Autonomia oraria: }&\dfrac{W_{f}}{W_i} = e^{-\dfrac{c\Delta t}{E}} 
\end{align*}

avendo effettuato i calcoli aerodinamici, si utilizza la reale efficienza di crociera della particolare configurazione. Come in precedenza, si suppone l'efficienza in loiter aumentata di un fattore 1/0.866 rispetto a quella in crociera, ed il consumo specifico \textit{c} pari a 0.5 $kg/(kg\cdot h)$ nella fase di crociera e 0.4 $kg/(kg\cdot h)$ in loiter.

Si ricava dunque la \textit{fuel fraction}, moltiplicando il rapporto di peso in ogni fase della missione (aumentata del 5\% per contingenza), e quindi il peso di carburante necessario per la missione, moltiplicando la \textit{fuel fraction} per il peso al decollo.

Un altro fondamentale parametro di prestazione è il \textit{block fuel}, ovvero il carburante consumato per una missione standard, senza diversione.

\vspace{1cm}
\subsection{Costi}
Una volta giunto a convergenza il calcolo del peso al decollo, si calcolano i costi associati alla particolare configurazione, altri parametri importanti per valutare la bontà di una configurazione. 

Per valutare l'efficienza del velivolo è stato calcolato il PREE (Payload-Range Efficiency). È un parametro che tiene conto sia dell'efficienza energetica che della capacità di trasporto del velivolo, ed è definito come:
\[
PREE = \dfrac{W_{payload} \cdot \text{Range}}{FED \cdot W_{block\,fuel}}
\]
dove $FED$ (Fuel Energy Density) è il contenuto energetico per unità di massa del carburante, pari a 12000 $Wh/kg$.
\vspace{0.5cm}
\newline Il parametro principale per valutare l'efficienza economica del velivolo sono i \textit{Total Operating Cost}, divisi in una parte indipendente dal design del velivolo ovvero i \textit{Indirect Operating Cost} ed una parte direttamente dipendente dalla bontà del design, i \textit{Direct Operating Cost}. Essi rappresentano il costo in dollari per chilometro per passeggero, ed ovviamente è preferibile avere un valore basso:
\[DOC = \dfrac{TC}{N_{\text{passeggeri}}\cdot V_B}\]
dove $TC$ è il costo totale orario, $N_{\text{passeggeri}}$ il numero di passeggeri e $V_B$ la velocità commerciale, a sua volta definita come il rapporto tra il range e il \textit{block time}, ovvero il tempo di esecuzione di una missione standard (tempo in volo + tempo di operazioni a terra). In questa fase preliminare del design, è approssimato a \[B_t = \dfrac{Range}{V_{\text{cruise}}}+0.25\,h\]


A sua volta, il costo totale orario è composto da tre contributi principali: i costi di volo, i costi di manutenzione ed i costi di proprietà. Essendo costi orari, sono tutti rapportati al block time.
\vspace{0.5cm}
\newline Il primo contributo è il \textbf{costo di volo}, che comprende il costo dell'equipaggio, del carburante e delle tasse. Il costo orario dell'equipaggio è stato calcolato considerando uno stipendio di 315 \$/h per ciascun pilota e 77 \$/h per ogni assistente di volo, valori velidi al 1986 ed aggiornati secondo l'inflazione. Il costo del carburante è calcolato come il prodotto tra il costo del combustibile e il consumo orario; quest'ultimo è ottenuto dividendo la massa di block fuel per il block time.

Per quanto riguarda le tasse, sono state considerate tre componenti: la tassa di atterraggio, proporzionale alla massa al decollo (7 \$/tonnellata); la tassa di navigazione, calcolata in funzione del range e della massa al decollo (in tonnellate) secondo la formula:
\[
\text{Tassa di Navigazione} = 0.4 \cdot \text{Range} \cdot \sqrt{\dfrac{WTO}{50}}
\]
ed infine la tassa di terra, proporzionale al payload (93 \$/tonnellata). Anche questi valori sono stati aggiornati secondo l'inflazione.
\vspace{0.5cm}
\newline Il secondo contributo principale è il \textbf{costo di manutenzione}, diviso tra struttura e motori. È possibile suddividere ancora queste componenti in costo della manodopera e costo dei materiali:
\[\text{Costo di Manutenzione} = A_{LABOUR}+A_{MATERIAL}+N_{prop}\cdot (E_{LABOUR} + E_{MATERIAL})\cdot \left(\dfrac{Flight_t+1.3}{B_t}\right)\]
dove $A$ indica i costi associati alla struttura (\textit{Airframe}) ed $E$ quelli associati ai motori (\textit{Engine}).

Il costo orario della manodopera è calcolato come:
\[
A_{labour} = \left(0.09 \cdot W_{airframe} + 6.7 - \dfrac{350}{W_{airframe} + 75}\right)\cdot\dfrac{0.8 + 0.68 \cdot Flight_{t}}{Block_{t}} \cdot cost_{man\,hour}
\]
dove $W_{airframe}$ è la massa della struttura in tonnellate (ottenuta sottraendo la massa dei motori dalla massa a vuoto), $Flight_{t}$ e $Block_{t}$ sono rispettivamente il tempo di volo e il block time in ore, e $c_{man\,hour}$ è il costo orario della manodopera.

Il costo orario dei materiali è invece calcolato come:
\[
A_{material} = \dfrac{4.2 + 2.2 \cdot Flight_{t}}{Block_{t}} \cdot (ADP - N_{prop} \cdot BEP)
\]
dove $ADP$ (\textit{Aircraft Delivery Price}) è il prezzo di consegna del velivolo, funzione della velocità di crociera, della massa a vuoto e del numero di motori; $BEP$ (Base Engine Price) è invece il prezzo base del motore, calcolato statisticamente in funzione della spinta.

Per i motori, il costo della manodopera è calcolato come:
\[
E_{labour} = 0.21 \cdot cost_{man\,hour} \cdot C_1 \cdot C_3 \cdot (1 + T \cdot 10^{-3})^{0.4}
\]
dove $C_1$ è un coefficiente funzione del rapporto di bypass, posto pari a 12 (valore realistico per i motori turbofan più efficienti esistenti), $C_3$ dipende dal numero di stadi del compressore, ipotizzati dodici, e $T$ è la spinta installata in tonnellate.

Il costo dei materiali è invece:
\[
E_{material} = 2.56 \cdot (1 + T \cdot 10^{-3})^{0.8} \cdot C_1 \cdot (C_2 + C_3)
\]
dove $C_2$ è un coefficiente funzione del rapporto di compressione totale del motore, posto pari a 32.

Il costo totale orario è quindi la somma di tutti questi contributi.

Infine, è stato valutato l'impatto ambientale del velivolo sotto forma di emissioni di CO\textsubscript{2}, considerando che per ogni kg di carburante bruciato vengono prodotti 3.16 kg di CO\textsubscript{2}.

\section{Risultati}
Delle 2160 possibili configurazioni solo 411 rispettano il vincolo del carico alare massimo calcolato dal \textit{matching chart} e dell'apertura alare massima di 36 metri. 

Si sono raccolti i risultati nella seguente \textbf{matrice delle configurazioni}:

\begin{landscape}
    \begin{table}[H]
        \centering
        \renewcommand{\arraystretch}{1.2}
        \adjustbox{max width=0.855 \paperheight, max height=\paperheight}{%
        \begin{tabular}{|c|c|c|c|c|c|c|c|c|c|c|c|c|c|c|c|c|c|c|c|c|c|c|c|c|c|c|c|c|c|c|c|c|c|c|c|c|c|c|}
            \hline
            \multicolumn{14}{|c|}{\textbf{Geometria}} & 
            \multicolumn{12}{c|}{\textbf{Performance}} & 
            \multicolumn{8}{c|}{\textbf{Pesi}} & 
            \multicolumn{5}{c|}{\textbf{Costi}} \\
            \hline
            \textbf{W/S} & \textbf{S} & \textbf{S}$\mathbf{_{wet}}$ & \textbf{S}$\mathbf{_{vert}}$ & \textbf{S}$\mathbf{_{orizz}}$ & \textbf{b} & $\mathbf{\lambda}$ & \textbf{c}$\mathbf{_{root}}$ & \textbf{c}$\mathbf{_{tip}}$ & \textbf{MAC} & \textbf{AR} & \textbf{t/c} & $\mathbf{\Lambda_{25}}$ & \textbf{V}$\mathbf{_{fuel}}$ [l] & \textbf{M} & \textbf{W}$\mathbf{_{fuel}}$ & \textbf{W}$\mathbf{_{block\,fuel}}$ & \textbf{E}$\mathbf{_{cruise}}$ & \textbf{C}$\mathbf{_{D}}$ & \textbf{C}$\mathbf{_{D_0}}$ & \textbf{C}$\mathbf{_{D_i}}$ & \textbf{C}$\mathbf{_{D_W}}$ & \textbf{C}$\mathbf{_{L_{cruise}}}$ & \textbf{C}$\mathbf{_{L_{max}}}$ & \textbf{T} [kg] & \textbf{T/W} & \textbf{MTOW} & \textbf{OEW} & \textbf{Wing} & \textbf{Tail} & \textbf{Land.Gear} & \textbf{Propulsion} & \textbf{Fuel Sys.} & \textbf{Hydraulics} & \textbf{ADP} & \textbf{Flight Cost} & \textbf{Maint. Cost} & \textbf{DOC} & \textbf{PREE} \\
            \hline
            \boldmath $600$ & \boldmath $141.37$ & \boldmath $289.66$ & \boldmath $25.4$ & \boldmath $35.3$ & \boldmath $35.7$ & \boldmath $0.23$ & \boldmath $7.28$ & \boldmath $1.67$ & \boldmath $5.06$ & \boldmath $9$ & \boldmath $0.10$ & \boldmath $25$ & \boldmath $24543.67$ & \boldmath $0.760$ & \boldmath $19634.94$ & \boldmath $16843.52$ & \boldmath $16.1$ & \boldmath $0.0345$ & \boldmath $0.0217$ & \boldmath $0.0123$ & \textbf{\num{5.10e-4}} & \boldmath $0.556$ & \boldmath $2.49$ & \boldmath $25956.45$ & \boldmath $0.31$ & \boldmath $84831.72$ & \boldmath $43916.78$ & \boldmath $8568.67$ & \boldmath $1492.85$ & \boldmath $3427.50$ & \boldmath $5449.99$ & \boldmath $710.67$ & \boldmath $1493.39$ & \boldmath $84.3$ & \boldmath $4953.12$ & \boldmath $1036.18$ & \boldmath $0.0425$ & \boldmath $0.474$ \\            
             600 & 141.54 & 290.01 & 25.5 & 35.4 & 35.7 & 0.27 & 7.06 & 1.91 & 4.97 & 9 & 0.10 & 25 & 24546.18 & 0.760 & 19636.95 & 16845.58 & 16.1 & 0.0345 & 0.0217 & 0.0123 & \num{5.10e-4} & 0.556 & 2.49 & 25928.03 & 0.31 & 84934.61 & 44017.66 & 8667.21 & 1494.66 & 3432.02 & 5443.74 & 711.08 & 1495.17 & 84.4 & 4953.98 & 1037.21 & 0.0425 & 0.474 \\
             600 & 141.74 & 290.43 & 25.5 & 35.4 & 35.7 & 0.31 & 6.84 & 2.12 & 4.90 & 9 & 0.10 & 25 & 24566.49 & 0.760 & 19653.19 & 16859.70 & 16.1 & 0.0344 & 0.0217 & 0.0123 & \num{5.10e-4} & 0.556 & 2.49 & 25931.01 & 0.300 & 85057.66 & 44124.47 & 8763.06 & 1496.82 & 3437.42 & 5444.39 & 711.57 & 1497.31 & 84.6 & 4956.40 & 1038.50 & 0.0425 & 0.473 \\
             600 & 140.50 & 287.89 & 25.3 & 35.1 & 35.6 & 0.23 & 7.26 & 1.67 & 5.05 & 9 & 0.10 & 20 & 24591.73 & 0.760 & 19673.38 & 16874.10 & 15.9 & 0.0349 & 0.0217 & 0.0119 & \num{1.32e-3} & 0.556 & 2.58 & 25328.60 & 0.300 & 84322.97 & 43369.59 & 8228.04 & 1483.74 & 3404.71 & 5311.91 & 684.50 & 1484.40 & 83.4 & 4953.75 & 1025.96 & 0.0424 & 0.473 \\
             600 & 140.65 & 288.19 & 25.3 & 35.2 & 35.6 & 0.27 & 7.04 & 1.90 & 4.96 & 9 & 0.10 & 20 & 24592.23 & 0.760 & 19673.78 & 16874.74 & 15.9 & 0.0349 & 0.0217 & 0.0119 & \num{1.32e-3} & 0.556 & 2.58 & 25297.90 & 0.300 & 84413.34 & 43459.56 & 8317.42 & 1485.29 & 3408.58 & 5305.16 & 684.85 & 1485.92 & 83.5 & 4954.36 & 1026.85 & 0.0424 & 0.473 \\
             600 & 140.83 & 288.57 & 25.4 & 35.2 & 35.6 & 0.31 & 6.83 & 2.12 & 4.88 & 9 & 0.10 & 20 & 24610.70 & 0.760 & 19688.56 & 16887.58 & 15.9 & 0.0348 & 0.0217 & 0.0118 & \num{1.32e-3} & 0.556 & 2.58 & 25298.26 & 0.300 & 84524.83 & 43556.27 & 8404.75 & 1487.24 & 3413.46 & 5305.24 & 685.28 & 1487.85 & 83.6 & 4956.56 & 1028.00 & 0.0425 & 0.473 \\
             650 & 128.97 & 264.25 & 23.2 & 32.2 & 34.1 & 0.23 & 7 & 1.61 & 4.87 & 9 & 0.10 & 20 & 24794.23 & 0.760 & 19835.39 & 17009.05 & 15.6 & 0.0386 & 0.0229 & 0.0140 & \num{1.68e-3} & 0.602 & 2.58 & 26249.12 & 0.31 & 83849.78 & 42734.39 & 7686.06 & 1361.91 & 3383.88 & 5514.43 & 657.04 & 1364.19 & 83.0 & 4967.05 & 1023.80 & 0.0425 & 0.469 \\
             650 & 129.13 & 264.59 & 23.2 & 32.3 & 34.1 & 0.27 & 6.78 & 1.83 & 4.78 & 9 & 0.10 & 20 & 24802.77 & 0.760 & 19842.21 & 17015.17 & 15.6 & 0.0385 & 0.0229 & 0.0139 & \num{1.68e-3} & 0.602 & 2.58 & 26282.77 & 0.31 & 83959.77 & 42837.55 & 7773.07 & 1363.66 & 3388.61 & 5521.84 & 657.44 & 1365.92 & 83.1 & 4968.43 & 1025.23 & 0.0425 & 0.469 \\
             650 & 129.31 & 264.97 & 23.3 & 32.3 & 34.1 & 0.31 & 6.58 & 2.04 & 4.71 & 9 & 0.10 & 20 & 24825.11 & 0.760 & 19860.09 & 17030.64 & 15.6 & 0.0385 & 0.0229 & 0.0139 & \num{1.68e-3} & 0.602 & 2.58 & 263203 & 0.31 & 84079.36 & 42939.27 & 7856.89 & 1365.59 & 3393.84 & 5530.05 & 657.89 & 1367.82 & 83.2 & 4970.99 & 1026.66 & 0.0426 & 0.469 \\
            \\
             600 & 169.28 & 351.25 & 30.5 & 42.3 & 34.4 & 0.31 & 8.52 & 2.64 & 6.10 & 7 & 0.15 & 20 & 42643.48 & 0.800 & 34114.78 & 29110.90 & 9.09 & 0.0551 & 0.0217 & 0.0123 & 0.0211 & 0.501 & 2.58 & 34630.50 & 0.340 & 101590.79 & 46196.01 & 7630.11 & 1787.62 & 4164.53 & 7377.87 & 663.57 & 1784.23 & 92.5 & 6764.69 & 1126.64 & 0.0533 & 0.274 \\
             650 & 156.04 & 323.78 & 28.1 & 39.0 & 33.0 & 0.27 & 8.49 & 2.29 & 5.98 & 7 & 0.15 & 20 & 43182.77 & 0.800 & 34546.22 & 29480.78 & 8.92 & 0.0609 & 0.0228 & 0.0145 & 0.0236 & 0.543 & 2.58 & 35932.21 & 0.350 & 101448.27 & 45622.06 & 7079.07 & 1647.80 & 4158.27 & 7670.19 & 638.23 & 1646.28 & 92.2 & 6810.24 & 1126.46 & 0.0536 & 0.271 \\
             650 & 156.12 & 323.95 & 28.1 & 39.0 & 33.1 & 0.31 & 8.23 & 2.55 & 5.89 & 7 & 0.15 & 20 & 43194.55 & 0.800 & 34555.64 & 29488.78 & 8.92 & 0.0609 & 0.0228 & 0.0145 & 0.0236 & 0.543 & 2.58 & 35908.87 & 0.350 & 101500.34 & 45664.70 & 7122.84 & 1648.64 & 4160.55 & 7664.95 & 638.39 & 1647.11 & 92.3 & 6811.55 & 1126.85 & 0.0536 & 0.271 \\
             650 & 156.01 & 323.72 & 28.1 & 39.0 & 33.0 & 0.23 & 8.76 & 2.02 & 6.09 & 7 & 0.15 & 20 & 43195.78 & 0.800 & 34556.62 & 29489.73 & 8.91 & 0.0610 & 0.0228 & 0.0146 & 0.0236 & 0.543 & 2.58 & 35998.50 & 0.350 & 101428.14 & 45591.51 & 7035.13 & 1647.48 & 4157.39 & 7685.10 & 638.17 & 1645.96 & 92.2 & 6811.24 & 1126.43 & 0.0536 & 0.271 \\
             600 & 172.48 & 357.90 & 31.1 & 43.1 & 34.8 & 0.27 & 8.86 & 2.39 & 6.25 & 7 & 0.15 & 25 & 44102.69 & 0.820 & 35282.15 & 30070.24 & 8.70 & 0.0548 & 0.0216 & 0.0116 & 0.0217 & 0.477 & 2.49 & 36044.32 & 0.350 & 103512.72 & 46950.57 & 7901.28 & 1821.42 & 4249.04 & 7695.41 & 693.09 & 1805.27 & 95.1 & 7023.96 & 1151.02 & 0.0539 & 0.265 \\
             600 & 172.41 & 357.75 & 31.0 & 43.1 & 34.7 & 0.23 & 9.15 & 2.10 & 6.36 & 7 & 0.15 & 25 & 44103.42 & 0.820 & 35282.73 & 30070.80 & 8.70 & 0.0549 & 0.0216 & 0.0117 & 0.0217 & 0.477 & 2.49 & 36101.28 & 0.350 & 103470.25 & 46907.51 & 7848.85 & 1820.68 & 4247.18 & 7708.22 & 692.96 & 1804.53 & 95.1 & 7023.78 & 1150.79 & 0.0539 & 0.265 \\
             600 & 172.64 & 358.23 & 31.1 & 43.2 & 34.8 & 0.31 & 8.60 & 2.66 & 6.15 & 7 & 0.15 & 25 & 44134.43 & 0.820 & 35307.54 & 30091.85 & 8.71 & 0.0548 & 0.0216 & 0.0116 & 0.0217 & 0.477 & 2.49 & 36036.78 & 0.350 & 103598.57 & 47011.03 & 7955.40 & 1823.12 & 4253.29 & 7693.71 & 693.40 & 1806.96 & 95.2 & 7027.25 & 1151.71 & 0.0540 & 0.265 \\
             650 & 170.93 & 354.69 & 30.8 & 42.7 & 34.6 & 0.27 & 8.83 & 2.38 & 6.22 & 7 & 0.15 & 20 & 52512.74 & 0.820 & 42010.19 & 35834.59 & 7.52 & 0.0688 & 0.0222 & 0.0131 & 0.0334 & 0.517 & 2.58 & 39243.87 & 0.350 & 111127.69 & 47837.49 & 7773.93 & 1805.10 & 4584.36 & 8417.16 & 666.66 & 1801.48 & 97.4 & 7809.24 & 1178.04 & 0.0593 & 0.223 \\
             650 & 170.89 & 354.59 & 30.8 & 42.7 & 34.6 & 0.23 & 9.11 & 2.10 & 6.34 & 7 & 0.15 & 20 & 52518.39 & 0.820 & 42014.71 & 35838.61 & 7.51 & 0.0688 & 0.0222 & 0.0132 & 0.0334 & 0.517 & 2.58 & 39313.12 & 0.350 & 111095.77 & 47801.06 & 7724.27 & 1804.59 & 4582.97 & 8432.83 & 666.57 & 1800.97 & 97.3 & 7809.57 & 1177.94 & 0.0593 & 0.223 \\
             650 & 171.04 & 354.90 & 30.8 & 42.8 & 34.6 & 0.31 & 8.56 & 2.65 & 6.13 & 7 & 0.15 & 20 & 52534.81 & 0.820 & 42027.85 & 35849.57 & 7.52 & 0.0688 & 0.0222 & 0.0131 & 0.0334 & 0.517 & 2.58 & 39221.87 & 0.350 & 111195.23 & 47887.39 & 7823.44 & 1806.20 & 4587.33 & 8412.18 & 666.85 & 1802.56 & 97.4 & 7811.56 & 1178.53 & 0.0593 & 0.223 \\
            \hline
        \end{tabular}%
        }
        \caption{Esempio di tabella ridimensionata}
    \end{table}
\end{landscape}
  

Per poter scegliere la configurazione migliore si sono tenute in considerazione le più importanti \textbf{figure di merito}, ovvero parametri fisici dell'aereo come massa massima al decollo, spinta installata ed efficienza aerodinamica, e parametri economici come i costi operativi, il PREE e l'ADP.

Si è notato che il block fuel è un ottimo indicatore dei costi. Risulta quindi uno dei parametri più importanti per la scelta della configurazione migliore, e quello su cui inevitabilmente si è concentrata l'analisi dei risultati.
\begin{figure}[H]
    \centering
    \includegraphics[width=\textwidth]{E1//Immagini/costi vs block fuel.png}
    \caption{Parametri di costo in funzione del block fuel}
\end{figure}

Un altro parametro direttamente collegato ai costi è il peso massimo al decollo.
\begin{figure}[H]
    \centering
    \includegraphics[width=\textwidth]{E1//Immagini/costi vs mtow.png}
    \caption{Parametri di costo in funzione del peso massimo al decollo}
\end{figure}

\noindent Si è dunque analizzato il legame tra block fuel e peso massimo al decollo, e si è notato che i due parametri sono legati circa linearmente. Si sono quindi creati degli \textit{scatter plot} per analizzare la dipendenza dalle variabili di design.

Si nota che i valori più bassi di block fuel e MTOW si hanno per: carico alare di 600 o 650 kg/m$^2$, aspect ratio di 9, angolo di freccia di $20\si{\degree}$ o $25\si{\degree}$, spessore del profilo del 10\% e Mach di crociera di 0.76.

\begin{figure}[H] \centering \includegraphics[width=\textwidth]{E1/Immagini/block fuel = f(WTO, W_S).png} \caption{Scatter plot di block fuel e MTOW raggruppati per carico alare} \end{figure} \begin{figure}[H] \centering \includegraphics[width=\textwidth]{E1/Immagini/block fuel = f(WTO, AR).png} \caption{Scatter plot di block fuel e MTOW raggruppati per aspect ratio} \end{figure} \begin{figure}[H] \centering \includegraphics[width=\textwidth]{E1/Immagini/block fuel = f(WTO, freccia).png} \caption{Scatter plot di block fuel e MTOW raggruppati per angolo di freccia} \end{figure} \begin{figure}[H] \centering \includegraphics[width=\textwidth]{E1/Immagini/block fuel = f(WTO, t_c).png} \caption{Scatter plot di block fuel e MTOW raggruppati per spessore del profilo} \end{figure} \begin{figure}[H] \centering \includegraphics[width=\textwidth]{E1/Immagini/block fuel = f(WTO, M).png} \caption{Scatter plot di block fuel e MTOW raggruppati per Mach di crociera} \end{figure}

