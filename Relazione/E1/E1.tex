\Chapter{Esercitazione 1}{Velivolo convenzionale}
\label{sec:E1}

Nella prima esercitazione si è progettato un nuovo velivolo convenzionale, di tipo trasporto passeggeri a medio raggio, come possibile successore dell'Airbus A320.

\section{Requisito}
Per definire il requisito del nuovo velivolo si è partiti innanzitutto dall'analisi statistica degli aerei commerciali dello stesso settore di maggior diffusione sul mercato odierno.

\begin{table}[H]
\centering
\footnotesize
\begin{tabular}{ccccccccccccc}
Model&Max&Length&Wing&Fuselage&Wing&Aspect&MTOW&Max&Fuel&OEW&Cruise&Range\\
&Seats&&Span&Length&Area&Ratio&&Payload&&&Speed&\\
&&[m]&[m]&[m]&$\text{[m}^2\text{]}$&&[t]&[t]&[l]&[t]&[km/h]&[km]\\
\toprule
\normalsize\textbf{Airbus}&&&&&&&&&&&& \\
\midrule
A220-300 & 160 & 38.71 & 35.1 & 3.5 & 112.3 & 8.89 & 70.9 & 18.7 & 21920 & 37.1 & 823 & 6300 \\
A318 & 136 & 31.45 & 34.1 & 3.95 & 122.4 & 7.43 & 59-68 & 15 & 24210 & 39.6 & 829 & 5700 \\
A319ceo & 160 & 33.84 & 34.1 & 3.95 & 122.6 & 7.41 & 75.5 & 17.7 & 24210 & 40.8 & 829 & 6850 \\
A319neo & 160 & 33.84 & 35.8 & 3.95 & 124 & 8.18 & 78.2 & 17.7 & 24210 & 42.6 & 833 & 6950 \\
A320ceo & 180 & 37.57 & 34.1 & 3.95 & 122.6 & 7.41 & 78 & 19.9 & 24210 & 42.6 & 829 & 6100 \\
\textbf{A320neo} & \textbf{195} & \textbf{37.57} & \textbf{35.8} & \textbf{3.95} & \textbf{124} & \textbf{8.18} & \textbf{79} & \textbf{20} & \textbf{24210} & \textbf{44.3} & \textbf{833} & \textbf{6500} \\
A321ceo & 220 & 44.51 & 34.1 & 3.95 & 122.6 & 7.41 & 89 & 22.8 & 23700 & 48 & 829 & 5900 \\
A321neo & 244 & 44.51 & 35.8 & 3.95 & 124 & 8.18 & 93.5 & 25.5 & 23500 & 50.1 & 833 & 6500 \\
A321XLR & 244 & 44.51 & 35.8 & 3.95 & 124 & 8.18 & 101 & 25.5 & 40000 & 50.1 & 833 & 8700 \\
\toprule
\normalsize\textbf{Boeing}&&&&&&&&&&&& \\
\midrule
B737-700 & 149 & 32.18 & 34.3 & 3.73 & 124.6 & 7.50 & 69.4 & 17 & 26000 & 37.6 & 838 & 6000 \\
B737-800 & 189 & 38.08 & 34.3 & 3.73 & 124.6 & 7.50 & 78.2 & 20.2 & 26000 & 41.4 & 838 & 5420 \\
B737-7 & 172 & 35.56 & 35.92 & 3.76 & 127 & 8.14 & 80 & -- & 25800 & -- & 839 & 7000 \\
B737-8 & 189 & 39.47 & 35.92 & 3.76 & 127 & 8.14 & 82.6 & -- & 25800 & -- & 839 & 6500 \\
B737-9 & 220 & 42.16 & 35.92 & 3.76 & 127 & 8.14 & 88.3 & -- & 25800 & -- & 839 & 6100 \\
B737-10 & 230 & 43.79 & 35.92 & 3.76 & 127 & 8.14 & 89.8 & -- & 25800 & -- & 839 & 5700 \\
\toprule
\normalsize\textbf{Comac}&&&&&&&&&&&& \\
\midrule
C919 & 192 & 38.9 & 35.8 & 3.96 & 129.1 & 7.85 & 75.1 & 18.9 & 24900 & 45.7 & 838 & 4140 \\
C919ER & 192 & 38.9 & 35.8 & 3.96 & 129.1 & 7.85 & 78.9 & 18.9 & 24900 & 45.7 & 838 & 5576 \\
\bottomrule
\end{tabular}
\caption{Dati di aerei dello stesso segmento.}
\end{table}

\noindent A partire dalla ricerca di mercato, si sono stilati una serie di requisiti per il nuovo velivolo.

\begin{table}[H]
\centering
\begin{tabular}{c|c}
Numero di passeggeri & 180 \\
Numero di piloti & 2 \\
Assistenti di volo & 4 \\
Velocità di crociera & 835 km/h \\
Range & 4500 km \\
Range di diversione & 200 nm \\
Numero di motori & 2 \\
Specific Fuel Consumption & 0.5 $kg/(kg\cdot h)$ \\
\end{tabular}
\caption{Requisiti del nuovo velivolo.}
\end{table}

Il numero di assistenti di volo è stato determinato a partire dal numero dei passeggeri, con la normativa EASA ORO.CC.100(b)(3) (un assistente ogni 50 passeggeri); lo Specific Fuel Consumption è leggermente migliorato rispetto all'Airbus A320neo, ipotizzando un avanzamento tecnologico che permette di migliorare l'efficienza dei propulsori.
\vspace{0.5cm}
\newline Un altro fondamentale requisito vincolante è che l'apertura alare sia compresa tra i 24 e i 36 metri, in modo da far rientrare il velivolo nella classe C della classificazione dei velivoli definita dall'ICAO.
\newpage
\section{Metodologia}
\vspace{1cm}
\subsection{Stima preliminare della massa massima al decollo}
Il primo parametro che si è stimato, fondamentale per il resto del progetto, è la massa massima al decollo. È possibile dividerla in quattro componenti: equipaggio, payload, carburante e massa a vuoto. 
\[m_{TO} = m_{crew}+m_{payload}+m_{fuel}+m_{empty}\]
È possibile riarrangiare la precedente equazione in modo da esplicitare la frazione di carburante e la frazione di massa a vuoto
\[m_{TO} = \dfrac{m_{crew}+m_{payload}}{1-\dfrac{m_{fuel}}{m_{TO}}-\dfrac{m_{empty}}{m_{TO}}}\]
Si sono quindi calcolate le diverse componenti della massa.

\subsubsection*{Massa equipaggio e payload}
In questa fase iniziale di progetto si è assunto che tutto il payload sia fatto dai passeggeri e dai loro bagagli, non considerando quindi eventuali merci aggiuntive.

Si è quindi calcolata la massa considerando la stima statistica elaborata da Roskam, secondo cui il peso di un singolo passeggero è 79.4 kg e il suo bagaglio, per un volo di corto-medio raggio, 13.6 kg, per un totale di \textbf{93.0 kg}. Si è assunto lo stesso peso per il personale di bordo.

In definitiva, la massa totale per le 186 persone a bordo (180 passeggeri + 6 crew) è pari a \textbf{17298 kg}.

\subsubsection*{Frazione di massa a vuoto}
È stata utilizzata la stima di Raymer, secondo cui
\[ \dfrac{m_{empty}}{m_{TO}} = A*m_{TO}^C\]
Dove \textit{A} e \textit{C} sono parametri costanti, dipendenti dal tipo di velivolo. Per un aereo da trasporto a getto valgono rispettivamente 0.97 e -0.06.

Si nota che questa frazione dipende dalla stessa massa al decollo; sarà quindi necessario effettuare un calcolo iterativo.

\subsubsection*{Frazione di carburante}
La quantità di carburante necessaria può essere stimata a partire dalla missione tipo del velivolo. 
\begin{figure}[H]
    \centering
    \includegraphics[width=0.7\linewidth]{E1//Immagini/missione.png}
    \caption{Missione tipica di un aereo commerciale.}
\end{figure}

È quindi possibile dividere la missione in più fasi, e per ogni fase stimare il rapporto di peso tra la fine e l'inizio della fase stessa. 

In particolare è possibile ricavare questa stima da dati statistici per tutte le fasi tranne tre, ovvero la crociera, la diversione (che si tratta a tutti gli effetti di una seconda crociera) e il loiter. Per queste tre fasi è necessario ricorrere alle equazioni di autonomia (chilometrica e oraria) ricavabili dalla meccanica del volo:
\begin{align}
\text{Autonomia chilometrica: }&\dfrac{W_{i+1}}{W_i} = e^{-\dfrac{c\Delta x}{VE}}
\label{eq:autonomia} \\
\text{Autonomia oraria: }&\dfrac{W_{i+1}}{W_i} = e^{-\dfrac{c\Delta t}{E}} 
\end{align}
dove c è lo \textit{Specific Fuel Consumption}, V la velocità ed E l'efficienza aerodinamica. In particolare per lo SFC del loiter si è usato il valore di 0.4 $kg/(kg\cdot h)$, inferiore rispetto allo 0.5 $kg/(kg\cdot h)$ della crociera. Inoltre, il range considerato in questo step è di 4500 km, migliorando di 500 km quello dell'A320neo a pieno carico.

Queste formule sono ricavate facendo una semplificazione: quella di assetto costante, quindi $C_L$ costante; ne consegue che durante la crociera l'aereo prende quota grazie alla riduzione di peso dovuta al consumo di carburante, per una condizione di cosiddetto \textit{cruise-climb}.

È stato quindi necessario ricavare il valore dell'efficienza: si è utilizzata la formula, di nuovo ricavata da Raymer, secondo cui
\[\left(\dfrac{L}{D}\right)_{max} = k_E \sqrt{AR_w}\]
dove $k_E$ è un coefficiente dipendente dal tipo di aeroplano, in questo caso pari a 15.5, ed $AR_w$ l'aspect ratio della superficie bagnata:
\[AR_w = \dfrac{AR}{\dfrac{S_{wet}}{S_{ref}}}\]
dove AR è l'aspect ratio dell'ala, in questa fase preliminare imposta a 9 (in linea con aerei già esistenti) e il rapporto di superfici al denominatore è ricavato da aerei simili e posto pari a 6.4.

Si è quindi ricavata un'efficienza massima pari a 18.4, realistica per un aereo molto efficiente. Questa è pari all'efficienza nella fase di loiter, mentre per la fase di crociera va ridotta di un fattore 0.866.

Si sono infine calcolati i rapporti di peso delle tre fasi:
\begin{table}[H]
\centering
\begin{tabular}{ccc}
$\left(\dfrac{W_f}{W_i}\right)_{cruise} = 0.8443$ ; & $\left(\dfrac{W_f}{W_i}\right)_{diversion} = 0.974$ ; & $\left(\dfrac{W_f}{W_i}\right)_{loiter} = 0.989$ .
\end{tabular}
\end{table}

Come anticipato, il rapporto di peso delle altre fasi si ricava da dati statistici:
\begin{table}[H]
\centering
\begin{tabular}{ccc}
$\left(\dfrac{W_f}{W_i}\right)_{take off} = 0.970$ ; & $\left(\dfrac{W_f}{W_i}\right)_{climb} = 0.985$ ; & $\left(\dfrac{W_f}{W_i}\right)_{descent} = 0.995$ ;
\end{tabular}
\end{table}
\begin{table}[H]
\centering
\begin{tabular}{cc}
$\left(\dfrac{W_f}{W_i}\right)_{diversion\,climb} = 1$ ; & $\left(\dfrac{W_f}{W_i}\right)_{diversion\,descent} = 1$ .
\end{tabular}
\end{table}

Avendo infine ricavato il rapporto di pesi per tutte le fasi, è possibile calcolare il rapporto di peso tra la fine e l'inizio della missione:

\begin{equation}
    \left(\dfrac{W_f}{W_i}\right)_{mission} = \dfrac{W_{final}}{W_{TO}} = \prod_{i=1}^{n}\left(\dfrac{W_{i+1}}{W_i}\right) = 0.7734
    \label{eq:w_final}
\end{equation}

Il complemento ad uno di questo valore è la frazione di carburante dell'aereo. A questo valore va aggiunto il 5\% di riserva:
\begin{equation}
    \frac{m_{fuel}}{m_{TO}} = 1.05\left(1-\dfrac{m_{final}}{m_{TO}}\right) = \textbf{0.238}
    \label{eq:frazione_carburante}
\end{equation}

\subsubsection*{Peso al decollo}
Ricavati tutti gli elementi, è stato infine calcolato il peso al decollo tramite un calcolo iterativo su MATLAB. Il codice converge dopo 6 iterazioni, a partire da una stima iniziale di 60 tonnellate e con una tolleranza sulla differenza relativa di $10^{-4}$, su un valore di \textbf{65657 kg}. È un valore sostanzialmente inferiore di quello dell'A320neo di riferimento, giustificato dal minor numero di passeggeri e dalla mancanza di trasporto merci.

\subsubsection{Analisi di trade-off}
A partire da questo risultato, sono state effettuate analisi di trade-off al variare di alcuni parametri.

\begin{table}[H]
    \centering
    \begin{tabular}{c|c}
    \toprule
    Parametro & Range di valori \\
    \midrule
        Range & 3500 - 5500 km \\
        Payload &  160 - 200 passeggeri\\
        Specific Fuel Consumption & 0.4 - 0.6 $kg/(kg\cdot h)$ \\
        Aspect Ratio & 8 - 10 \\
         \bottomrule
    \end{tabular}
    \caption{Parametri per l'analisi di trade-off}
\end{table}

I valori ottenuti sono illustrati nei plot seguenti.

\begin{figure}[H]
    \centering
    \begin{subfigure}[b]{0.48\textwidth}
        \includegraphics[width=\textwidth]{E1/Immagini/tradeoff_range.png}
        \caption{Massa al decollo al variare del range}
    \end{subfigure}
    \hfill
    \begin{subfigure}[b]{0.48\textwidth}
        \includegraphics[width=\textwidth]{E1/Immagini/tradeoff_payload.png}
        \caption{Massa al decollo al variare del payload}
    \end{subfigure}
    \medskip
    \begin{subfigure}[b]{0.48\textwidth}
        \includegraphics[width=\textwidth]{E1/Immagini/tradeoff_sfc.png}
        \caption{Massa al decollo al variare dello SFC}
    \end{subfigure}
    \hfill
    \begin{subfigure}[b]{0.48\textwidth}
        \includegraphics[width=\textwidth]{E1/Immagini/tradeoff_AR.png}
        \caption{Massa al decollo al variare dell'aspect ratio}
    \end{subfigure}
    \medskip
    \caption{Risultato delle analisi di trade-off}
\end{figure}

Come previsto la massa aumenta con l'aumentare del range, del payload e del consumo, mentre diminuisce all'aumentare dell'aspect ratio in quanto questo parametro è direttamente collegato all'efficienza aerodinamica.

\vspace{1cm}
\subsection{Diagramma Payload-Range preliminare}
Basandosi sui dati ricavati dal calcolo delle masse, si è ricavato il diagramma payload-range per l'aeroplano. Esso è definito da quattro punti:
\begin{itemize}
    \item \textbf{Punto A}: corrisponde al payload massimo e senza carburante a bordo;
    \item \textbf{Punto B}: corrisponde al \textit{punto armonico}, ovvero il range massimo a payload massimo. È il punto di progetto dell'aeroplano, quindi con range e payload definiti da requisito;
    \item \textbf{Punto C}: corrisponde al range massimo raggiungibile con pieno carico di carburante, sacrificando del payload per rientrare nel peso massimo al decollo;
    \item \textbf{Punto D}: corrisponde al range massimo teorico raggiungibile dall'aeroplano vuoto, ovvero senza payload e con carburante massimo.
\end{itemize}

\noindent Il primo punto si ricava banalmente, il range è di 0 km ed il payload quello massimo dei passeggeri.

\noindent Il secondo punto è ugualmente semplice da ricavare, in quanto corrisponde al payload massimo ed il range di requisito.

\medskip

\noindent Per calcolare il terzo punto si sono dovute fare delle assunzioni. Innanzitutto la capacità dei serbatoi, fissata a \textbf{25000 l} in base alla capacità di aerei della stessa categoria; inoltre è stato necessario assumere la densità del carburante: si è utilizzato il valore di \textbf{0.785 kg/l}, coerente con la letteratura per il carburante \textit{Jet-A} e lo stesso valore utilizzato da Airbus per i propri calcoli del range.

In questo modo è stato possibile ricavare la massa del pieno carico di carburante, pari a 19625 kg, e la frazione di carburante con massa massima al decollo, pari a 0.260. Si è quindi potuto calcolare il rapporto della massa a fine missione semplicemente invertendo l'equazione \ref{eq:frazione_carburante}:
\[\frac{m_{final}}{m_{TO}} = 1-\dfrac{1}{1.05}\dfrac{m_{fuel}}{m_{TO}}\]

Si è quindi calcolato il range della crociera a partire dall'equazione dell'autonomia chilometrica (\ref{eq:autonomia}), considerando che la frazione di massa finale è pari al prodotto delle frazioni di massa in ogni fase della missione (come da equazione \ref{eq:w_final}) e che tutte le altre frazioni restano uguali al punto armonico:
\[
\text{Range}_C = -\ln\left(\dfrac{\dfrac{m_{final}}{m_{TO}}}{\dfrac{W_2}{W_{TO}} \cdot \dfrac{W_3}{W_2} \cdot \dfrac{W_5}{W_4} \cdot \dfrac{W_8}{W_7} \cdot \dfrac{W_{10}}{W_9}}\right) \cdot \dfrac{v_{cruise} \cdot E_{cruise}}{SFC_{cruise}} = 6575\,km
\]

\noindent Il payload corrisponde invece a:
\[
m_{payload_C} = m_{TO} - (m_{empty} + m_{fuel_{max}} + m_{crew}) = 12739 kg
\]
pari a 137 passeggeri.

\medskip


\noindent Infine, per il calcolo del range dell'ultimo punto si è calcolata la massa al decollo, pari a
\[
m_{TO_D} = m_{empty} + m_{fuel_{max}} + m_{crew} = 57102 kg
\]
dove è stata considerata una crew di 2, ovvero solo i piloti, in quanto non essendoci passeggeri non c'è necessità di personale di bordo.

Successivamente il calcolo è uguale a quello per il punto C: si è calcolata la frazione di carburante (pari a 0.344) ed il range della crociera alla stessa maniera: il valore che si ricava è di $9356\,km$, con payload ovviamente pari a 0.

\begin{figure}[H]
    \centering
    \includegraphics[width=0.76\linewidth]{E1//Immagini/payload-range.png}
    \caption{Diagramma payload-range dell'aeroplano.}
\end{figure}

\subsubsection*{Famiglia di aerei}
Si è ipotizzato di voler modificare l'aeroplano in modo da creare una famiglia, sulla falsa riga di quella dell'Airbus A320 con A319 e A321, quindi semplicemente modificando la fusoliera in modo da accogliere diversi numeri di passeggeri. In particolare, si è ipotizzata una configurazione a 150 passeggeri ed un'altra a 210 passeggeri. È stata inoltre considerata una configurazione a 210 passeggeri e range esteso, ipotizzando che grazie alla fusoliera più grande ci sia spazio per dei serbatoi aggiuntivi, in particolare fino a 35000 litri di carburante. 

In maniera del tutto analoga per il caso a 180 passeggeri - con l'unica differenza nel numero di assistenti di bordo in quanto da normativa ne bastano 3 per il primo caso e ne servono 5 per il secondo - si sono calcolate le masse ed i range per l'intera famiglia.

\begin{figure}[H]
    \centering
    \includegraphics[width=0.95\linewidth]{E1//Immagini/payload-range_family.png}
    \caption{Diagramma payload-range della famiglia di aerei.}
\end{figure}

Si notano ovvie differenze di range, dovute innanzitutto alla differenza di payload, ma anche alla differenza di peso a vuoto.

\begin{table}[H]
    \centering
    \begin{tabular}{c|c|c|c|c|c}
    \toprule
         Passeggeri & $range_C$ [km] & $range_D$ [km] & $m_{TO}$ [kg] & $m_{empty}$ [kg] & $m_{empty}/m_{TO}$ \\
         \midrule
         150 & 8524 & 10825 & 55754 & 28072 & 0.504 \\
         180 & 6575 & 9356 & 65657 & 32735 & 0.499 \\
         210 & 5239 & 8210 & 75418 & 37291 & 0.495 \\
         210 (ER) & 8997 & 11377 & 75418 & 37291 & 0.495 \\
         \bottomrule
    \end{tabular}
    \caption{Valori di range e masse per la famiglia di aerei.}
\end{table}

\vspace{1cm}
\subsection{Matching Chart}
A partire dai risultati preliminari è stato creato il \textit{matching chart}. Si tratta di un grafico in cui sono riportate le curve della spinta necessaria (rapportata al peso), in diverse fasi di volo, in funzione del carico alare. Se ne ricava la spinta minima da installare sul veivolo in modo da soddisfare la normativa.

\subsubsection*{Carico alare massimo}
Il primo parametro da determinare è il carico alare massimo, che si ricava dalla formula della portanza alla velocità minima, ovvero quella di stallo:
\[
L = W = \dfrac{1}{2}\rho V^2 S C_L \implies \left(\dfrac{W}{S}\right)_{max} = \dfrac{1}{2}\rho V_{stall}^2 C_{L_{max}}
\]
La velocità di stallo è imposta a 115 nodi, in linea con gli aerei della stessa categoria, ed aumentata di un fattore 1.2 per simulare le condizioni di volo a bassa velocità.

Per il coefficiente di portanza massimo è stato necessario scegliere un profilo alare: il \textbf{NASA SC(2)-0610}, ovvero un profilo supercritico adatto al volo transonico. Tramite il tool online \textit{airfoiltools.com} ne è stato ricavato il coefficiente di portanza massimo, pari a 1.7. 
\begin{figure}[H]
    \centering
    \includegraphics[width=0.95\linewidth]{E1//Immagini/profilo alare.png}
    \caption{Profilo NASA SC(2)-0610}
\end{figure}
Dal coefficiente di portanza del profilo si è passati a quello dell'ala:
\[C_{L_{max}}^{3D} = 0.9\,C_{L_{max}}^{2D}\cos{\Lambda_{25}}\]
dove $\Lambda_{25}$ è l'angolo di freccia al quarto di corda, in questa fase preliminare fissato a 25 gradi.

Questo valore del $C_L$ va aumentato del contributo dato dai dispositivi di ipersostentazione, in questo caso slat al bordo d'attacco e Fowler flap al bordo di fuga che producono un incremento $\Delta C_L^{2D} = 1.55$. Per calcolare il contributo dato all'ala si usa la seguente formula:
\[\Delta C_L^{3D} = 0.92\,\Delta C_L^{2D}\,\dfrac{S_{flapped}}{S}\cos{\Lambda_{25}}\]
Il valore $S_{flapped}$ rappresenta l'intera superficie alare su cui sono installati gli ipersostentatori; il rapporto con la superficie alare è stato calcolato pari a 0.85 da misure dell'ala dell'Airbus A320.
\newline
\noindent Il coefficiente di portanza massimo così calcolato è pari a 2.49.

In definitiva, si ricava un carico alare massimo di $715.5 \,kg/m^2$.
\subsubsection*{Decollo}
Per calcolare la spinta richiesta al decollo si è usato l'approccio semplificato elaborato da Raymer:
\[\left(\dfrac{T}{W}\right)_{take-off} = \dfrac{W/S}{TOP\,\sigma\,C_L^{take-off}}\]
dove: $\sigma$ è il rapporto tra la densità dell'aria alla quota dell'aeroporto e a livello del mare, in questo caso imposto a 1; come $C_L^{take-off}$ si è usato il coefficiente di portanza massimo, già ricavato per il calcolo del carico alare massimo; \textit{TOP} è il \textit{Take Off Parameter}, un parametro calcolato semi-empiricamente da Raymer e ricavato dal seguente grafico:
\begin{figure}[H]
    \centering
    \includegraphics[width=0.65\linewidth]{E1//Immagini/Take Off Parameter.png}
    \caption{Grafico del Take Off Parameter}
\end{figure}
Si entra nel grafico con la distanza di decollo, per cui è stata usata la \textit{Balanced Field Length} dell'A320 pari a 6900 ft e, intercettando la corretta curva relativa a due motori a getto, si legge il TOP pari a $165 \,\,lb/ft^2$, convertito nel sistema internazionale per effettuare i calcoli.

\subsubsection*{Salita}
L'equazione per il rapporto di spinta è ricavata dalla meccanica del volo:
\begin{equation}
    \dfrac{T}{W} = k_{OEI}\left(\frac{1}{2}\rho_{SL}\dfrac{V^2}{W/S}\left(C_D+\text{K}\left(\dfrac{W/S}{\rho_{SL}V^2}\cos{\gamma}\right)^2\right)+\sin{\gamma}\right)
    \label{eq:spinta_salita}
\end{equation}
dove: $ k_{OEI}$ è il parametro di \textit{One Engine Inoperative}, nel caso di aereo bimotore pari a 2; la densità dell'aria è approssimata a quella a livello del mare, anche se aumentando di quota diminuisce; K$\, = \frac{1}{\pi\,AR\,e}$ è il parametro per il calcolo della resistenza indotta $C_{D_i} = \text{K}C_L^2$, con il fattore di Oswald imposto pari a 0.8 in questa fase preliminare. \newline Il $C_{D_0}$ è stato imposto pari a 0.017 a partire da dati statistici per velivoli della stessa categoria.

\vspace{0.5cm}
Da normativa la fase di salita è divisa in tre segmenti, e il veivolo deve soddisfare i seguenti requisiti con un motore inoperativo:
\begin{table}[H]
    \centering
    \renewcommand{\arraystretch}{1.2}
    \begin{tabular}{c|c|c|c}
       & Primo segmento & Secondo segmento & Terzo segmento \\
    \toprule
        Comincia & fine pista & carrello retratto & configurazione finale \\
    \midrule
        Gradiente minimo $\gamma$ & 0.0\% & 2.4\% & 1.2\% \\
    \midrule
       Configurazione ipersostentatori  & decollo & decollo & retratti \\
    \midrule
       Carrello  & in retrazione & retratto & retratto \\
    \midrule
       Velocità & $V_2$ & $V_2$ & $1.25\,V_2$ \\
    \end{tabular}
\end{table}
\noindent La velocità $V_2$ è quella caratteristica del decollo, in questa fase posta pari a 1.2 volte la velocità di stallo.

\noindent Gli ipersostentatori aumentano il coefficiente di resistenza:
\[C_{D,flap} = 0.9\left(\dfrac{c_{flap}}{c}\right)^{1.38}\left(\dfrac{S_{w,flap}}{S}\right)\sin^2{\delta_{flap}}\]
dove i rapporti tra le corde e le superfici sono stati ricavati dai dati dell'ala dell'A320 e $\delta_{flap}$ rappresenta l'angolo di deflessione dei flap, in fase di decollo pari a $35\si{\degree}$.

Anche il carrello d'atterraggio aumenta il coefficiente di resistenza:
\[C_{D,LG} = \num{2.92e-3}\dfrac{W_{TO}^{0.785}}{S}\]
dove il valore del peso al decollo è quello ricavato dal calcolo preliminare, e il valore della superficie alare è ricavato da quello del carico alare.
\vspace{0.5cm}
Per rappresentare sul matching chart la spinta richiesta per la salita, è stata considerata la condizione più gravosa tra i tre segmenti, che corrisponde al secondo segmento in quanto è richiesto un gradiente maggiore.

\subsubsection*{Crociera}
Anche in questo caso l'equazione per la spinta richiesta è ricavata dalla meccanica del volo:
\[\dfrac{T}{W} = \dfrac{\dfrac{1}{2}\rho_{cruise}V^2}{W/S}\left(C_{D_0}+\text{K}\left(\dfrac{W/S}{\rho_{cruise}V^2}\right)^2\right)\]
dove la velocità e la quota di crociera sono, in questa fase, imposte da requisito rispettivamente a 835 km/h e a 33000 piedi, che con il modello dell'aria standard corrisponde ad una densità di 0.4097 kg/m$^3$.

Si è infine riferita la spinta richiesta al livello del mare per poterla paragonare con le altre fasi di volo:
\[\left(\dfrac{T}{W}\right)_{SL} = \dfrac{\left(\dfrac{T}{W}\right)_{cruise}}{\dfrac{\rho_{cruise}}{\rho_{SL}}}\]

\subsubsection*{Approach Climb}
È la condizione per cui si verifica un guasto ad un motore durante la fase di approccio (carrello retratto e flap a $20\si{\degree}$), ad una velocità pari a 1.41 volte la velocità di stallo. Il gradiente di salita richiesto è del 2.1\%. \newline L'equazione per ricavare il rapporto di spinta è esattamente la stessa per la fase di salita (equazione \ref{eq:spinta_salita}). 

\subsubsection*{Landing Climb}
È la condizione per cui si verifica un guasto ad un motore durante la fase di atterraggio (carrello esteso e flap a $35\si{\degree}$), ad una velocità pari a 1.23 volte la velocità di stallo. Il gradiente di salita richiesto è del 3.2\%. \newline L'equazione per ricavare il rapporto di spinta è esattamente la stessa per la fase di salita (equazione \ref{eq:spinta_salita}). 

\subsubsection*{Matching Chart}
\begin{figure}[H]
    \centering
    \includegraphics[width=\linewidth]{E1//Immagini/Matching Chart.png}
    \caption{Matching Chart dell'aeroplano.}
\end{figure}
Ricavata la spinta minima richiesta per ogni fase di volo, si può definire la spinta minima da installare sul velivolo, come la massima tra le spinte richieste in ogni fase. In generale è preferibile installare la spinta minore possibile per ridurre il peso del sistema propulsivo ed il consumo di carburante.

\vspace{1cm}
\subsection{Fusoliera}
La prima parte del velivolo da disegnare, in quanto uguale per tutte le diverse configurazioni aerodinamiche e propulsive, è la fusoliera. La sua funzione primaria è accogliere il payload (passeggeri, bagagli e cargo); da requisito deve poter ospitare 180 passeggeri e relativi bagagli, 4 assistenti di volo e 2 piloti. 
\subsubsection*{Diametro esterno}
Si è scelto, in questa fase di design preliminare, di mantenere una sezione circolare per tutta la lunghezza della fusoliera, nonostante nella realtà ciò è vero solo per i velivoli \textit{wide body}. \newline Si è ipotizzata una larghezza del singolo sedile pari a 0.54 metri, standard per sedili di classe economica. Con sei sedili per fila e la larghezza del corridoio pari a 0.50 metri, nel range di valori soliti per i velivoli \textit{narrow body}, si ricava il diametro interno pari a 3.74 metri; si posizionano i sedili in modo tale che la loro larghezza massima corrisponda al diametro massimo della sezione. \newline Si è inoltre considerato uno spessore ulteriore per la struttura stessa della fusoliera e per la pannellatura interna, che porta il diametro esterno della fusoliera a \textbf{4.10 metri}.
\subsubsection*{Lunghezza}
Il numero di passeggeri permette di stimare la lunghezza della cabina, ad esclusione della sezione di coda e di prua: si è assunto un \textit{pitch} tra i sedili pari a 0.72 metri, e calcolato il numero di file semplicemente dividendo il numero di passeggeri per il numero di sedili a fila, ovvero sei, ricavando così 30 file e quindi una lunghezza di 21.60 metri. 

In realtà facendo così si sovrastima leggermente la lunghezza della cabina, in quanto alcune file rientrano nella parte di coda e di prua, ma ciò permette di considerare anche lo spazio in più necessario in cabina per elementi come i sedili degli assistenti di volo e le toilette.\newline La lunghezza delle sezioni di poppa e di prua è stata stimata grazie a formule statistiche: la parte di poppa pari a 2.5 volte il diametro esterno, quella di prua pari a 1.5 volte il diametro esterno: le loro lunghezze sono rispettivamente 6.15 e 10.25 metri. \newline La lunghezza totale della fusoliera è quindi di \textbf{38 metri}.

\subsubsection{Stiva}
Per calcolare il volume della stiva si è divisa la fusoliera lasciando il giusto spazio in cabina per il comfort dei passeggeri, ed inoltre si è omesso lo spazio per ospitare il carrello d'atterraggio, la \textit{wingbox} in corrispondenza dell'ala e altri sistemi. Si è così ottenuto un volume di $\mathbf{43 \,\,\textbf{m}^3}$. 

Successivamente, si è stimato il volume dei bagagli dei passeggeri, in base a dati statistici, pari a 0.113 $m^3$ per bagaglio, ottenendo quindi 20.34 $m^3$ del volume della stiva occupato dai bagagli dei passeggeri.

Il rimanente spazio è stato riservato per le merci: in particolare per container \textbf{LD3/46(W)}, standard per questa categoria di aerei, il cui volume è di 4.04 $m^3$. Si sono quindi potuti inserire 5 container, per un totale di 20.20 $m^3$.

\vspace{1cm}
\subsection{Analisi di sensibilità}
A partire dai risultati preliminari, è stata impostata un'analisi di sensibilità in modo da ricavare diverse configurazioni. In particolare, sono stati scelti alcuni parametri aerodinamici come variabili di design, e fatti variare entro un range ragionevole:

\begin{table}[H]
    \centering
    \renewcommand{\arraystretch}{1.2}
    \begin{tabular}{|c|c|c|}
    \hline
        \rowcolor{grey}
        Variabile  & Simbolo & Valori \\
    \hline
        Carico Alare & \textit{W/S} & $[550\, 600\, 650\, 700]\,\, kg/m^2$ \\
    \hline
        Aspect Ratio & AR & $[7\, 8\, 9\, 10\, 11]$ \\
    \hline
       Rapporto Spessore Alare - Corda  & \textit{t/c} & $[0.10\, 0.12\, 0.15]$ \\
    \hline
       Mach di Crociera  & M & $[0.76\, 0.80\, 0.82]$ \\
    \hline
       Angolo di Freccia al quarto di corda & $\Lambda_{25}$ & $[20\, 25\, 30\, 35]\,\,\si{\degree} $ \\
    \hline
       Rapporto di Rastremazione & $\lambda$ & $[0.23\, 0.27\, 0.31]$ \\
    \hline
    \end{tabular}
    \caption{Variabili di design}
\end{table}

Sono stati quindi calcolati i parametri di peso, aerodinamici e di prestazione per ogni combinazione delle variabili di design. Per fare ciò si è impostato su MATLAB un ciclo di convergenza sul peso massimo al decollo, in quanto è un parametro che influenza e viene influenzato da tutti gli altri. Si è imposta una tolleranza sulla differenza relativa di 25 kg e si è partiti dal valore iniziale calcolato in via preliminare, ovvero 65657 kg.
\vspace{0.5cm}
\newline Il primo step in ogni fase del ciclo è quello di calcolare mediante il matching chart il valore della spinta da installare per quella configurazione, funzione del carico alare. I passaggi successivi sono il calcolo dei parametri aerodinamici, la stima dei pesi, delle prestazioni in missione ed infine, una volta giunto a convergenza il ciclo, i costi associati alla particolare configurazione.

\subsection{Aerodinamica}
L'obiettivo di questa sezione del calcolo è quello di trovare il coefficiente di resistenza del velivolo. I tre principali contributi alla resistenza sono la resistenza a portanza nulla $C_{D_0}$, a cui contribuisce l'intero velivolo, la resistenza indotta $C_{D_i}$ e la resistenza d'onda $C_{D_W}$, generate dal funzionamento dell'ala:\[C_D=C_{D_0}+ C_{D_i} + C_{D_W}\]

\paragraph*{\texorpdfstring{$\mathbf{C_{D_0}}$}{CD0}}
    
    È calcolato come somma dei contributi del $C_{D_0}$ di ala, fusoliera, impennaggi di coda e gondole motore. È stato calcolato secondo la formula derivata da Raymer:
    \[C_{D_0} = \dfrac{\sum_i c_{F_i}\cdot FF_i\cdot Q_i\cdot S_{W_i}}{S}\]
    Dove: $c_F$ è il coefficiente di attrito equivalente, $FF$ il fattore di forma, $Q$ il fattore di interferenza e $S_W$ la superficie bagnata, mentre $S$ è la superficie alare. Essa è calcolata ad ogni iterazione del ciclo come rapporto tra il peso al decollo e il carico alare; l'apertura alare $b$ è calcolata come radice quadrata del prodotto tra $Aspect Ratio$ e superficie alare.

    Il coefficiente d'attrito equivalente per un flusso turbolento è pari a:
    \[ c_{F_{TURB}} = \frac{0.455}{\left[ \log_{10}(Re) \right]^{2.58} \left[ 1 + 0.144 \cdot M^2 \right]^{0.65}} \]
    dove $M$ è il Mach di crociera, variabile di design, e $Re$ il numero di Reynolds calcolato nelle condizioni di crociera: velocità pari al Mach di crociera per velocità del suono alla quota di crociera (pari a $316.01\,m/s$) e viscosità dinamica calcolata tramite le relazioni dell'$International\,\,Standard\,\,Air$, pari a \num{1.591e-5} $kg/(m\cdot s)$; la lunghezza caratteristica $L$ è differente per ogni componente.
    
    A seguire i parametri per il calcolo del $C_{D_0}$ di ogni componente:
        \begin{itemize}
            \item \textbf{Ala}:
            \begin{table}[H]
                \centering
                \renewcommand{\arraystretch}{1.2}
                \begin{tabular}{c|c}
                    L & $MAC = \dfrac{2}{3}c_{root}\dfrac{1+\lambda+\lambda^2}{1+\lambda}$  ;  $c_{root} = \dfrac{S}{\dfrac{b-d_{fus}}{2}\left(1+\lambda\right)}$ \\
                    \midrule \\
                    FF & $\left[ 1 + \dfrac{0.6}{(x/c)_m} \cdot \left( \dfrac{t}{c} \right) + 100 \cdot \left( \dfrac{t}{c} \right)^4 \right] \cdot \left[ 1.34 \cdot M_{cruise}^{0.18} \cdot (\cos \Lambda_{25})^{0.28} \right]$ ; $(x/c)_m \approx 0.5$ \\
                    \midrule \\
                    Q & 1.0 \\
                    \midrule \\
                    $S_W$ & $S\left[1.997+0.52\left(\dfrac{t}{c}\right)\right]$
                \end{tabular}
             \end{table}
            \item \textbf{Fusoliera}:
            \begin{table}[H]
                \centering
                \renewcommand{\arraystretch}{1.2}
                \begin{tabular}{c|c}
                    L & $l_{fus} = 38.0 \,m$ \\
                    \midrule \\
                    FF & $\left(1+\dfrac{60}{\left(l_{fus}/d_{fus}\right)^3}+\dfrac{\left(l_{fus}/d_{fus}\right)}{400}\right)$ \\
                    \midrule \\
                    Q & 1.0 \\
                    \midrule \\
                    $S_W$ & $\pi\left(\dfrac{d_{fus}^2}{2}+d_{fus}\left(l_{fus}-\dfrac{d_{fus}}{2}-l_T\right)\right)+\left(d_{fus}+\pi\right)*\dfrac{l_t}{2}$
                \end{tabular}
            \end{table}
            dove la superficie della fusoliera è stata divisa in: una semisfera in prua, un cilindro centrale e una superficie trapezoidale per la sezione di coda.

            \item \textbf{Equilibratore}:
            \begin{table}[H]
                \centering
                \renewcommand{\arraystretch}{1.2}
                \begin{tabular}{c|c}
                    L & $b_{orizz} = \sqrt{AR_{orizz}S_{orizz}}$ \\
                    \midrule \\
                    FF & $\left[ 1 + \dfrac{0.6}{(x/c)_m} \cdot \left( \dfrac{t}{c} \right)_{orizz} + 100 \cdot \left( \dfrac{t}{c} \right)_{orizz}^4 \right] \cdot \left[ 1.34 \cdot M^{0.18}_{cruise} \cdot (\cos \Lambda_{25_{orizz}})^{0.28} \right]$ ; $(x/c)_m \approx 0.5$ \\
                    \midrule \\
                    Q & 1.05 \\
                    \midrule \\
                    $S_W$ & $S\left[1.997+0.52\left(\dfrac{t}{c}\right)_{orizz}\right]$
                \end{tabular}
             \end{table}
             Per i parametri geometrici dell'equilibratore si è fatto riferimento a dati statistici per velivoli della stessa categoria:
             \begin{table}[H]
                \centering
                \renewcommand{\arraystretch}{1.2}
                \begin{tabular}{c|c}
                    $S_{orizz}$ & $0.25\, S_{wing}$\\
                    $AR_{orizz}$ & 5.0 \\
                    $\Lambda_{25_{orizz}}$ & $29\si{\degree}$ \\
                    $\left(\dfrac{t}{c}\right)_{orizz}$ & 0.10
                \end{tabular}
            \end{table} 

            \item \textbf{Timone}:
            \begin{table}[H]
                \centering
                \renewcommand{\arraystretch}{1.2}
                \begin{tabular}{c|c}
                    L & $b_{vert} = \sqrt{AR_{vert}S_{vert}}$ \\
                    \midrule \\
                    FF & $\left[ 1 + \dfrac{0.6}{(x/c)_m} \cdot \left( \dfrac{t}{c} \right)_{vert} + 100 \cdot \left( \dfrac{t}{c} \right)_{vert}^4 \right] \cdot \left[ 1.34 \cdot M^{0.18}_{cruise} \cdot (\cos \Lambda_{25_{vert}})^{0.28} \right]$ ; $(x/c)_m \approx 0.5$ \\
                    \midrule \\
                    Q & 1.05 \\
                    \midrule \\
                    $S_W$ & $S\left[1.997+0.52\left(\dfrac{t}{c}\right)_{vert}\right]$
                \end{tabular}
             \end{table}
             Anche i parametri geometrici del timone sono stati ricavati da dati statistici per velivoli della stessa categoria:
             \begin{table}[H]
                \centering
                \renewcommand{\arraystretch}{1.2}
                \begin{tabular}{c|c}
                    $S_{vert}$ & $0.18 \,S_{wing}$\\
                    $AR_{vert}$ & 1.8 \\
                    $\Lambda_{25_{vert}}$ & $34\si{\degree}$ \\
                    $\left(\dfrac{t}{c}\right)_{vert}$ & 0.12
                \end{tabular}
            \end{table} 

            \item \textbf{Gondole motore}:
            \begin{table}[H]
                \centering
                \renewcommand{\arraystretch}{1.2}
                \begin{tabular}{c|c}
                    L & $l_{nac} = 0.07 \sqrt{T/2}$  \\
                    \midrule \\
                    FF & $1+\dfrac{60}{f_{nac}^3}+\dfrac{f_{nac}}{400}$ ; $f_{nac} = \dfrac{l_{nac}}{\sqrt{(4/\pi)A_{nac}}}$ \\
                    \midrule \\
                    Q & 1.0 \\
                    \midrule \\
                    $S_W$ & $\pi d_{nac}l_{nac}$ ; $d_{nac} = 0.04 \sqrt{T/2}$
                \end{tabular}
            \end{table}
            dove la lunghezza ed il diametro della nacella è stata ricavata da una formula empirica, in cui la spinta è espressa in libbre ed il risultato è in piedi, opportunamente convertiti. Il fattore di forma è stato considerato simile a quello della fusoliera.
        \end{itemize}
        \vspace{0.6cm}

        \paragraph*{\texorpdfstring{$\mathbf{C_{D_i}}$}{CDi}}
    
    È calcolato come: \[C_{D_i} = K C_L^2\]
    dove K è il parametro adimensionale: \[K = \dfrac{1}{\pi AR\, e}\]
    A sua volta, il Fattore di Oswald $e$ è ricavato da una formula statistica interpolativa, dipendente dal rapporto di rastremazione: \[e = \dfrac{\cos(\Lambda_{25})}{1 + AR \cdot f_\lambda}\] 
    dove $f_\lambda$ è calcolato come: \[f_\lambda = 0.0524 \lambda^4 - 0.15 \lambda^3 + 0.1659 \lambda^2 - 0.0706 \lambda + 0.0119\]
    Infine, il coefficiente di portanza è calcolato direttamente dall'equazione della portanza in condizioni di crociera: \[C_L = \dfrac{2\,\,W/S}{\rho_{cruise} V_{cruise}^2}\]

    \paragraph*{\texorpdfstring{$\mathbf{C_{D_W}}$}{CDW}}
    La resistenza d'onda è direttamente legata all'angolo di freccia, che riduce il Mach del flusso incidente l'ala ed aumenta il Mach critico, a cui si iniziano a verificare fenomeni supersonici sul profilo. Esso è calcolato come: \begin{align*} 
        M_{dd} = \dfrac{0.90}{\cos \Lambda_{25}} &- \dfrac{(t/c)}{\cos^2 \Lambda_{25}} - \dfrac{0.10 \,C_L}{\cos^3 \Lambda_{25}} \\ \\
        M_{crit} = M_{dd} &- \left( \dfrac{0.1}{80} \right)^{1/3} 
    \end{align*}

    Se il Mach di crociera è minore del Mach critico, la resistenza d'onda è pari a zero; altrimenti, essa è calcolata come: \[C_{D_W} = 20 \cdot \left(M_{cruise} - M_{crit}\right)^4\]

Una volta ricavato il coefficiente di resistenza, è possibile calcolare l'efficienza aerodinamica in crociera, un parametro importante per la valutazione delle prestazioni del velivolo: \[E = \dfrac{L}{W} = \dfrac{C_L}{C_D}\]

\vspace{0.6cm}

\subsection{Stima dei Pesi}

La stima dei pesi rappresenta una parte fondamentale nella progettazione preliminare del velivolo, in quanto consente di definire i valori di massa richiesti per soddisfare i requisiti operativi e strutturali. In questa sezione, si dettagliano i principali componenti della massa operativa a vuoto (OEW, \textit{Operating Empty Weight}) e le relative dipendenze dai parametri di progetto. Tutte le formule sono ricavate empiricamente, sulla base di dati statistici di velivoli già esistenti, da Raymer; di conseguenza, tutti i parametri sono espressi, nelle equazioni, nel sistema imperiale.
\paragraph*{Ala}

Il peso dell'ala dipende da diversi parametri geometrici e strutturali, sintetizzati nell'indice geometrico dell'ala ($I_w$):

\begin{equation*}
I_w = \frac{n_{ult} \cdot AR^{1.5} \cdot \sqrt{\dfrac{W_{\text{zf}}}{W_{\text{TO}}}} \cdot (1 + 2\lambda) \cdot W/S \cdot S^{1.5} \cdot 10^{-6}}{\left(t/c\right) \cdot \cos^2 \Lambda_{25} \cdot (1 + \lambda)}
\end{equation*}

Dove:
\begin{itemize}
    \item $n_{\text{ult}}$: \textit{ultimate load factor}, fissato a 3.75;
    \item $W_{\text{zf}}$: peso a zero carburante $W_{\text{zf}} = W_{\text{TO}} \cdot (1 - f_{\text{fuel}})$, con la frazione di carburante $f_{\text{fuel}} = \dfrac{W_{\text{fuel}}}{W_{\text{TO}}}$ calcolata inizialmente nel calcolo preliminare del peso del velivolo, poi ricalcolata ad ogni step del ciclo di convergenza;
    \item $W_{\text{TO}}$: peso massimo al decollo.
\end{itemize}

In base al valore di $S$, il peso dell'ala viene calcolato come:

\begin{equation*}
W_{\text{wing}} = \begin{cases} 
    (0.93 \cdot I_w + 6.44 \cdot S + 390) & \text{se } S \geq 900 \ \text{ft}^2, \\
    (4.24 \cdot I_w + 0.57 \cdot S) & \text{altrimenti}.
\end{cases}
\end{equation*}

\paragraph*{Coda}

Il peso della coda convenzionale, incluse sia le superfici orizzontali e che quella verticale, è calcolato come:

\begin{equation*}
W_{\text{tail}} = 5.03 \cdot (S_{\text{orizz}} + S_{\text{vert}})
\end{equation*}

\paragraph*{Fusoliera}

Il peso della fusoliera dipende dalle sue dimensioni:

\begin{equation*}
W_{\text{fus}} = 1.35 \cdot \left( L_{\text{fus}} \cdot D_{\text{fus}} \right)^{1.28}
\end{equation*}

Dove:
\begin{itemize}
    \item $L_{\text{fus}}$: lunghezza della fusoliera;
    \item $D_{\text{fus}}$: diametro esterno della fusoliera.
\end{itemize}

\paragraph*{Carrello di atterraggio}

Il peso del carrello di atterraggio è calcolato considerando quattro contributi principali, tutti funzione del peso massimo al decollo ($W_{\text{TO}}$):

\begin{equation*}
W_{\text{LG}} = W_{\text{LG, strutt}} + W_{\text{LG, freni}} + W_{\text{LG, pneumatici}} + W_{\text{LG, controllo}}
\end{equation*}

I contributi individuali sono determinati come segue:
\begin{align*}
        W_{\text{LG, strutt}} &= W_{\text{LG, param}} \cdot \left( 0.45 + 23.1 \cdot 10^{-8} \cdot W_{\text{TO}} \right) \\
        W_{\text{LG, freni}} &= W_{\text{LG, param}} \cdot \left( 0.268 - 8.12 \cdot 10^{-8} \cdot W_{\text{TO}} \right) \\
        W_{\text{LG, pneumatici}} &= W_{\text{LG, param}} \cdot \left( 0.152 - 8.38 \cdot 10^{-8} \cdot W_{\text{TO}} \right) \\
        W_{\text{LG, controllo}} &= W_{\text{LG, param}} \cdot \left( 0.13 - 6.56 \cdot 10^{-8} \cdot W_{\text{TO}} \right) \\
\end{align*}

Il parametro $W_{\text{LG, param}}$ è definito come:

\begin{equation*}
W_{\text{LG, param}} = 0.044 \cdot W_{\text{TO}} - 672
\end{equation*}

\paragraph*{Propulsione}

Il peso del sistema propulsivo comprende motori e gondole:

\begin{equation*}
W_{\text{nac}} = \frac{1}{4} \cdot N_{\text{prop}} \cdot D_{\text{nac}} \cdot L_{\text{nac}} \cdot \left( T_{\text{engine}} \right)^{0.36}
\end{equation*}
dove $N_{\text{prop}}$ è il numero di motori, pari a due, e $T_{\text{engine}}$ la spinta del singolo motore, quindi quella totale installata divisa per il numero di motori.
\begin{equation*}
W_{\text{engine}} = \frac{\left(T/W\right) \cdot W_{\text{TO}}}{5.5}
\end{equation*}

\begin{equation*}
W_{\text{prop}} = W_{\text{nac}} + W_{\text{engine}}
\end{equation*}

\paragraph*{Sistemi}

Il peso dei sistemi include i seguenti contributi:

\begin{itemize}
    \item \textbf{Sistema carburante:} \begin{equation*} W_{\text{fuelsys}} = 2.71 \cdot \left( \frac{b}{\cos \Lambda_{25}} \cdot N_{\text{serbatoi}} \right)^{0.956} \end{equation*}
    con il numero di serbatoi imposto pari a sei, ovvero due in ogni semiala, uno centrale in fusoliera ed un \textit{trim tank} nell'equilibratore.
    \item \textbf{Sistema idraulico:} \begin{equation*} W_{\text{hydraulic}} = \begin{cases} 
        45 + 1.318 \cdot S_{\text{ref}} & \text{se } S_{\text{ref}} \leq 3000 \ \text{ft}^2, \\
        18.7 \cdot S_{\text{ref}}^{0.712} - 1620 & \text{altrimenti}.
    \end{cases} \end{equation*}
    con 
    \begin{equation*}
        S_{\text{ref}} = S_{\text{wing}} + 1.44\cdot(S_{\text{orizz}}+S_{\text{vert}})
    \end{equation*}
    \item \textbf{Sistema elettrico:} \begin{equation*} W_{\text{elec}} = 16.2 \cdot N_{\text{passeggeri}} + 110 \end{equation*}
    \item \textbf{Sistema pneumatico:} \begin{equation*} W_{\text{pneumatic}} = 26.2 \cdot N_{\text{passeggeri}}^{0.944} \end{equation*}
    \item \textbf{Sistema anti-ghiaccio}, senza anti-ghiaccio sulle superfici di coda: \begin{equation*} W_{\text{anti-ice}} = 0.120 \cdot S_{\text{wing}} \end{equation*}
    \item \textbf{Strumentazione} per la manetta, per il carburante, e altra: \begin{equation*} W_{\text{instr}} = \left( 0.00145 \cdot T_{\text{engine}} + 30 \right) \cdot N_{\text{prop}} + \left( 0.00714 \cdot V_{\text{fuel}} + 34 \right) + (1.872 \cdot N_{\text{passeggeri}} + 128) \end{equation*}
    \item \textbf{Avionica:} \begin{equation*} W_{\text{avionics}} = 2.8 \cdot N_{\text{pax}} + 2320 \end{equation*}
    \item \textbf{Sistema motore:} \begin{equation*} W_{\text{engine sys}} = 133 \cdot N_{\text{prop}} \end{equation*}
\end{itemize}

\paragraph*{Arredi e Servizi}

Il peso degli arredi e dei servizi per i passeggeri è calcolato come:

\begin{equation*}
W_{\text{furn}} = 118.4 \cdot N_{\text{passeggeri}} - 4190
\end{equation*}

Il peso dei servizi include contributi come catering e assistenza ai passeggeri:

\begin{equation*}
W_{\text{services}} = 2.529 \cdot N_{\text{passeggeri}} \cdot \left(\frac{\text{Range}}{M_{\text{cruise}}}\right)^{0.225}
\end{equation*}

\paragraph*{Equipaggio}

Il peso dell'equipaggio è calcolato considerando i due piloti ed i quattro assistenti di volo:

\begin{equation*}
W_{\text{crew}} = 2 \cdot 225 + 4 \cdot 155
\end{equation*}

\paragraph*{OEW totale}

Il peso operativo a vuoto è infine calcolato sommando tutti i contributi:

\begin{equation*} 
    \begin{split} 
        \text{OEW} = &\, W_{\text{wing}} + W_{\text{tail}} + W_{\text{fus}} + W_{\text{LG}} + W_{\text{prop}} + W_{\text{fuelsys}} + \\
         &\, W_{\text{hydraulic}} + W_{\text{elec}} + W_{\text{pneumatic}} + W_{\text{anti-ice}} + W_{\text{instr}} + \\
         &\, W_{\text{avionics}} + W_{\text{engine sys}} + W_{\text{furn}} + W_{\text{services}} + W_{\text{crew}}
    \end{split} 
\end{equation*}

\paragraph*{Payload}
Il carico pagante del velivolo è composto da passeggeri e relativi bagagli, e merci. 

Come anticipato nel calcolo del peso preliminare, il peso di un singolo passeggero è 79.4 kg e il suo bagaglio, per un volo di corto-medio raggio, 13.6 kg; ciò porta ad una massa totale dei passeggeri di $\mathbf{16740 \textbf{ kg}}$.

Calcolando il volume della stiva dell'aereo, si è stimato che possa contenere cinque container \textbf{LD3/46(W)}, standard per questa categoria di aerei, il cui peso lordo massimo è di 1135 kg, per un totale del peso delle merci di $\mathbf{5675 \textbf{ kg}}$. 

Il peso totale del payload si attesta quindi a $\mathbf{22415 \textbf{ kg}}$.

\subsection{Prestazioni in Missione}
L'ultimo step del ciclo è ricalcolare le prestazioni in missione, in particolare il consumo di carburante. Per ciò, si riutilizzano le formule di Breguet già usate nel calcolo preliminare, ma stavolta con la "vera" efficienza aerodinamica. si ricalcolano in particolare le differenze di peso (quindi i consumi di carburante) in crociera, loiter e diversione. 
\begin{align*}
    \text{Autonomia chilometrica: }&\dfrac{W_{f}}{W_i} = e^{-\dfrac{c\Delta x}{VE}}\\
    \text{Autonomia oraria: }&\dfrac{W_{f}}{W_i} = e^{-\dfrac{c\Delta t}{E}} 
\end{align*}

avendo effettuato i calcoli aerodinamici, si utilizza la reale efficienza di crociera della particolare configurazione. Come in precedenza, si suppone l'efficienza in loiter aumentata di un fattore 1/0.866 rispetto a quella in crociera, ed il consumo specifico \textit{c} pari a 0.5 $kg/(kg\cdot h)$ nella fase di crociera e 0.4 $kg/(kg\cdot h)$ in loiter.

Si ricava dunque la \textit{fuel fraction}, moltiplicando il rapporto di peso in ogni fase della missione (aumentata del 5\% per contingenza), e quindi il peso di carburante necessario per la missione, moltiplicando la \textit{fuel fraction} per il peso al decollo.

Un altro fondamentale parametro di prestazione è il \textit{block fuel}, ovvero il carburante consumato per una missione standard, senza diversione.

\vspace{1cm}
\subsection{Costi}
Una volta giunto a convergenza il calcolo del peso al decollo, si calcolano i costi associati alla particolare configurazione, altri parametri importanti per valutare la bontà di una configurazione. 

Per valutare l'efficienza del velivolo è stato calcolato il PREE (Payload-Range Efficiency). È un parametro che tiene conto sia dell'efficienza energetica che della capacità di trasporto del velivolo, ed è definito come:
\[
PREE = \dfrac{W_{payload} \cdot \text{Range}}{FED \cdot W_{block\,fuel}}
\]
dove $FED$ (Fuel Energy Density) è il contenuto energetico per unità di massa del carburante, pari a 12000 $Wh/kg$.
\vspace{0.5cm}
\newline Il parametro principale per valutare l'efficienza economica del velivolo sono i \textit{Total Operating Cost}, divisi in una parte indipendente dal design del velivolo ovvero i \textit{Indirect Operating Cost} ed una parte direttamente dipendente dalla bontà del design, i \textit{Direct Operating Cost}. Essi rappresentano il costo in dollari per chilometro per passeggero, ed ovviamente è preferibile avere un valore basso:
\[DOC = \dfrac{TC}{N_{\text{passeggeri}}\cdot V_B}\]
dove $TC$ è il costo totale orario, $N_{\text{passeggeri}}$ il numero di passeggeri e $V_B$ la velocità commerciale, a sua volta definita come il rapporto tra il range e il \textit{block time}, ovvero il tempo di esecuzione di una missione standard (tempo in volo + tempo di operazioni a terra). In questa fase preliminare del design, è approssimato a \[B_t = \dfrac{Range}{V_{\text{cruise}}}+0.25\,h\]


A sua volta, il costo totale orario è composto da tre contributi principali: i costi di volo, i costi di manutenzione ed i costi di proprietà. Essendo costi orari, sono tutti rapportati al block time.
\vspace{0.5cm}
\newline Il primo contributo è il \textbf{costo di volo}, che comprende il costo dell'equipaggio, del carburante e delle tasse. Il costo orario dell'equipaggio è stato calcolato considerando uno stipendio di 315 \$/h per ciascun pilota e 77 \$/h per ogni assistente di volo, valori velidi al 1986 ed aggiornati secondo l'inflazione. Il costo del carburante è calcolato come il prodotto tra il costo del combustibile e il consumo orario; quest'ultimo è ottenuto dividendo la massa di block fuel per il block time.

Per quanto riguarda le tasse, sono state considerate tre componenti: la tassa di atterraggio, proporzionale alla massa al decollo (7 \$/tonnellata); la tassa di navigazione, calcolata in funzione del range e della massa al decollo (in tonnellate) secondo la formula:
\[
\text{Tassa di Navigazione} = 0.4 \cdot \text{Range} \cdot \sqrt{\dfrac{WTO}{50}}
\]
ed infine la tassa di terra, proporzionale al payload (93 \$/tonnellata). Anche questi valori sono stati aggiornati secondo l'inflazione.
\vspace{0.5cm}
\newline Il secondo contributo principale è il \textbf{costo di manutenzione}, diviso tra struttura e motori. È possibile suddividere ancora queste componenti in costo della manodopera e costo dei materiali:
\[\text{Costo di Manutenzione} = A_{LABOUR}+A_{MATERIAL}+N_{prop}\cdot (E_{LABOUR} + E_{MATERIAL})\cdot \left(\dfrac{Flight_t+1.3}{B_t}\right)\]
dove $A$ indica i costi associati alla struttura (\textit{Airframe}) ed $E$ quelli associati ai motori (\textit{Engine}).

Il costo orario della manodopera è calcolato come:
\[
A_{labour} = \left(0.09 \cdot W_{airframe} + 6.7 - \dfrac{350}{W_{airframe} + 75}\right)\cdot\dfrac{0.8 + 0.68 \cdot Flight_{t}}{Block_{t}} \cdot cost_{man\,hour}
\]
dove $W_{airframe}$ è la massa della struttura in tonnellate (ottenuta sottraendo la massa dei motori dalla massa a vuoto), $Flight_{t}$ e $Block_{t}$ sono rispettivamente il tempo di volo e il block time in ore, e $c_{man\,hour}$ è il costo orario della manodopera.

Il costo orario dei materiali è invece calcolato come:
\[
A_{material} = \dfrac{4.2 + 2.2 \cdot Flight_{t}}{Block_{t}} \cdot (ADP - N_{prop} \cdot BEP)
\]
dove $ADP$ (\textit{Aircraft Delivery Price}) è il prezzo di consegna del velivolo, funzione della velocità di crociera, della massa a vuoto e del numero di motori; $BEP$ (Base Engine Price) è invece il prezzo base del motore, calcolato statisticamente in funzione della spinta.

Per i motori, il costo della manodopera è calcolato come:
\[
E_{labour} = 0.21 \cdot cost_{man\,hour} \cdot C_1 \cdot C_3 \cdot (1 + T \cdot 10^{-3})^{0.4}
\]
dove $C_1$ è un coefficiente funzione del rapporto di bypass, posto pari a 12 (valore realistico per i motori turbofan più efficienti esistenti), $C_3$ dipende dal numero di stadi del compressore, ipotizzati dodici, e $T$ è la spinta installata in tonnellate.

Il costo dei materiali è invece:
\[
E_{material} = 2.56 \cdot (1 + T \cdot 10^{-3})^{0.8} \cdot C_1 \cdot (C_2 + C_3)
\]
dove $C_2$ è un coefficiente funzione del rapporto di compressione totale del motore, posto pari a 32.

Il costo totale orario è quindi la somma di tutti questi contributi.

Infine, è stato valutato l'impatto ambientale del velivolo sotto forma di emissioni di CO\textsubscript{2}, considerando che per ogni kg di carburante bruciato vengono prodotti 3.16 kg di CO\textsubscript{2}.



\begin{landscape}
    \section{Risultati}
    Delle 2160 possibili combinazioni delle variabili di design, solo 412 configurazioni rispettano il vincolo del carico alare massimo calcolato dal \textit{matching chart} e dell'apertura alare massima di 36 metri. 

    Si sono raccolti i risultati nella seguente \textbf{matrice delle configurazioni}:
    \vspace*{\fill}
    \begin{table}[H]
        \centering
        \renewcommand{\arraystretch}{1.5}
        \setlength{\tabcolsep}{3pt}
        \adjustbox{max width=0.855 \paperheight, max height=\paperheight}{%
        \begin{tabular}{|c|c|c|c|c|c|c|c|c|c|c|c|c|c|c|c|c|c|c|c|c|c|c|c|c|c|c|c|c|c|c|c|c|c|c|c|c|c|c|c|}
            \hline
            \multicolumn{1}{|c|}{} & \multicolumn{14}{c|}{\textbf{Geometria}} & \multicolumn{12}{c|}{\textbf{Performance}} & \multicolumn{8}{c|}{\textbf{Pesi}} & \multicolumn{5}{c|}{\textbf{Costi}} \\
            \hline
            \textbf{n} & \cellcolor{yellow!30}\textbf{W/S} & \textbf{S} & \textbf{S}$\mathbf{_{wet}}$ & \textbf{S}$\mathbf{_{vert}}$ & \textbf{S}$\mathbf{_{orizz}}$ & \textbf{b} & \cellcolor{yellow!30}$\mathbf{\lambda}$ & \textbf{c}$\mathbf{_{root}}$ & \textbf{c}$\mathbf{_{tip}}$ & \textbf{MAC} & \cellcolor{yellow!30}\textbf{AR} & \cellcolor{yellow!30}\textbf{t/c} & \cellcolor{yellow!30}$\mathbf{\Lambda_{25}}$ & \textbf{V}$\mathbf{_{fuel}}$ [l] & \cellcolor{yellow!30}\textbf{M} & \textbf{W}$\mathbf{_{fuel}}$ & \textbf{W}$\mathbf{_{block\,fuel}}$ & \textbf{E}$\mathbf{_{cruise}}$ & \textbf{C}$\mathbf{_{D}}$ & \textbf{C}$\mathbf{_{D_0}}$ & \textbf{C}$\mathbf{_{D_i}}$ & \textbf{C}$\mathbf{_{D_W}}$ & \textbf{C}$\mathbf{_{L_{cruise}}}$ & \textbf{C}$\mathbf{_{L_{max}}}$ & \textbf{T} [kg] & \textbf{T/W} & \textbf{MTOW} & \textbf{OEW} & \textbf{Wing} & \textbf{Tail} & \textbf{Land.Gear} & \textbf{Prop.} & \textbf{Fuel Sys.} & \textbf{Hydraulic} & \textbf{ADP} & \textbf{Flight Cost} & \textbf{Maint. Cost} & \textbf{DOC} & \textbf{PREE} \\
            \hline
            1 & \cellcolor{yellow!30}650 & 129.49 & 265.32 & 23.31 & 32.37 & 35.98 & \cellcolor{yellow!30}0.23 & 6.60 & 1.52 & 4.59 & \cellcolor{yellow!30}10 & \cellcolor{yellow!30}0.10 & \cellcolor{yellow!30}20 & 24288 & \cellcolor{yellow!30}0.76 & 19430.75 & 16669.17 & 16.16 & 0.0373 & 0.0230 & 0.0126 & 0.0017 & 0.602 & 2.577 & 26354.73 & 0.31 & 84177.33 & 43466.58 & 8333.54 & 1367.39 & 3398.72 & 5537.69 & 692.30 & 1369.60 & 83.90 & 4928.41 & 1033.15 & 0.0423 & 0.479 \\
            \hline
            2 & \cellcolor{yellow!30}600 & 141.37 & 289.66 & 25.45 & 35.34 & 35.67 & \cellcolor{yellow!30}0.23 & 7.28 & 1.67 & 5.06 & \cellcolor{yellow!30}9 & \cellcolor{yellow!30}0.10 & \cellcolor{yellow!30}25 & 24544 & \cellcolor{yellow!30}0.76 & 19634.94 & 16843.52 & 16.09 & 0.0345 & 0.0217 & 0.0123 & 0.0005 & 0.556 & 2.485 & 25956.45 & 0.31 & 84831.72 & 43916.78 & 8568.67 & 1492.85 & 3427.50 & 5449.99 & 710.67 & 1493.39 & 84.32 & 4953.12 & 1036.18 & 0.0425 & 0.474 \\
            \hline
            3 & \cellcolor{yellow!30}600 & 141.54 & 290.01 & 25.48 & 35.38 & 35.69 & \cellcolor{yellow!30}0.27 & 7.06 & 1.91 & 4.97 & \cellcolor{yellow!30}9 & \cellcolor{yellow!30}0.10 & \cellcolor{yellow!30}25 & 24546 & \cellcolor{yellow!30}0.76 & 19636.95 & 16845.58 & 16.12 & 0.0345 & 0.0217 & 0.0123 & 0.0005 & 0.556 & 2.485 & 25928.03 & 0.31 & 84934.61 & 44017.66 & 8667.21 & 1494.66 & 3432.02 & 5443.74 & 711.08 & 1495.17 & 84.43 & 4953.98 & 1037.21 & 0.0425 & 0.474 \\
            \hline
            4 & \cellcolor{yellow!30}600 & 141.74 & 290.43 & 25.51 & 35.44 & 35.72 & \cellcolor{yellow!30}0.31 & 6.84 & 2.12 & 4.90 & \cellcolor{yellow!30}9 & \cellcolor{yellow!30}0.10 & \cellcolor{yellow!30}25 & 24566 & \cellcolor{yellow!30}0.76 & 19653.19 & 16859.70 & 16.13 & 0.0344 & 0.0217 & 0.0123 & 0.0005 & 0.556 & 2.485 & 25931.01 & 0.30 & 85057.66 & 44124.47 & 8763.06 & 1496.82 & 3437.42 & 5444.39 & 711.57 & 1497.31 & 84.57 & 4956.40 & 1038.50 & 0.0425 & 0.473 \\
            \hline
            \cellcolor{cyan!30}5 & \cellcolor{green!30}600 & \cellcolor{cyan!30}140.50 & \cellcolor{cyan!30}287.89 & \cellcolor{cyan!30}25.29 & \cellcolor{cyan!30}35.13 & \cellcolor{cyan!30}35.56 & \cellcolor{green!30}0.23 & \cellcolor{cyan!30}7.26 & \cellcolor{cyan!30}1.67 & \cellcolor{cyan!30}5.05 & \cellcolor{green!30}9 & \cellcolor{green!30}0.10 & \cellcolor{green!30}20 & \cellcolor{cyan!30}24592 & \cellcolor{green!30}0.76 & \cellcolor{cyan!30}19673.38 & \cellcolor{cyan!30}16874.10 & \cellcolor{cyan!30}15.91 & \cellcolor{cyan!30}0.0349 & \cellcolor{cyan!30}0.0217 & \cellcolor{cyan!30}0.0119 & \cellcolor{cyan!30}0.0013 & \cellcolor{cyan!30}0.556 & \cellcolor{cyan!30}2.577 & \cellcolor{cyan!30}25328.60 & \cellcolor{cyan!30}0.30 & \cellcolor{cyan!30}84322.97 & \cellcolor{cyan!30}43369.59 & \cellcolor{cyan!30}8228.04 & \cellcolor{cyan!30}1483.74 & \cellcolor{cyan!30}3404.71 & \cellcolor{cyan!30}5311.91 & \cellcolor{cyan!30}684.50 & \cellcolor{cyan!30}1484.40 & \cellcolor{cyan!30}83.40 & \cellcolor{cyan!30}4953.75 & \cellcolor{cyan!30}1025.96 & \cellcolor{cyan!30}0.0424 & \cellcolor{cyan!30}0.473 \\
            \hline
            6 & \cellcolor{yellow!30}600 & 140.65 & 288.19 & 25.32 & 35.16 & 35.58 & \cellcolor{yellow!30}0.27 & 7.04 & 1.90 & 4.96 & \cellcolor{yellow!30}9 & \cellcolor{yellow!30}0.10 & \cellcolor{yellow!30}20 & 24592 & \cellcolor{yellow!30}0.76 & 19673.78 & 16874.74 & 15.93 & 0.0349 & 0.0217 & 0.0119 & 0.0013 & 0.556 & 2.577 & 25297.90 & 0.30 & 84413.34 & 43459.56 & 8317.42 & 1485.29 & 3408.58 & 5305.16 & 684.85 & 1485.92 & 83.50 & 4954.36 & 1026.85 & 0.0424 & 0.473 \\
            \hline
            7 & \cellcolor{yellow!30}600 & 140.83 & 288.57 & 25.35 & 35.21 & 35.60 & \cellcolor{yellow!30}0.31 & 6.83 & 2.12 & 4.88 & \cellcolor{yellow!30}9 & \cellcolor{yellow!30}0.10 & \cellcolor{yellow!30}20 & 24611 & \cellcolor{yellow!30}0.76 & 19688.56 & 16887.58 & 15.95 & 0.0348 & 0.0217 & 0.0118 & 0.0013 & 0.556 & 2.577 & 25298.26 & 0.30 & 84524.83 & 43556.27 & 8404.75 & 1487.24 & 3413.46 & 5305.24 & 685.28 & 1487.85 & 83.62 & 4956.56 & 1028.00 & 0.0425 & 0.473 \\
            \hline
            8 & \cellcolor{yellow!30}650 & 128.97 & 264.25 & 23.21 & 32.24 & 34.07 & \cellcolor{yellow!30}0.23 & 7.00 & 1.61 & 4.87 & \cellcolor{yellow!30}9 & \cellcolor{yellow!30}0.10 & \cellcolor{yellow!30}20 & 24794 & \cellcolor{yellow!30}0.76 & 19835.39 & 17009.05 & 15.60 & 0.0386 & 0.0229 & 0.0140 & 0.0017 & 0.602 & 2.577 & 26249.12 & 0.31 & 83849.78 & 42734.39 & 7686.06 & 1361.91 & 3383.88 & 5514.43 & 657.04 & 1364.19 & 82.95 & 4967.05 & 1023.80 & 0.0425 & 0.469 \\
            \hline
            9 & \cellcolor{yellow!30}650 & 129.13 & 264.59 & 23.24 & 32.28 & 34.09 & \cellcolor{yellow!30}0.27 & 6.78 & 1.83 & 4.78 & \cellcolor{yellow!30}9 & \cellcolor{yellow!30}0.10 & \cellcolor{yellow!30}20 & 24803 & \cellcolor{yellow!30}0.76 & 19842.21 & 17015.17 & 15.62 & 0.0385 & 0.0229 & 0.0139 & 0.0017 & 0.602 & 2.577 & 26282.77 & 0.31 & 83959.77 & 42837.55 & 7773.07 & 1363.66 & 3388.61 & 5521.84 & 657.44 & 1365.92 & 83.09 & 4968.43 & 1025.23 & 0.0425 & 0.469 \\
            \hline
            10 & \cellcolor{yellow!30}650 & 129.31 & 264.97 & 23.28 & 32.33 & 34.12 & \cellcolor{yellow!30}0.31 & 6.58 & 2.04 & 4.71 & \cellcolor{yellow!30}9 & \cellcolor{yellow!30}0.10 & \cellcolor{yellow!30}20 & 24825 & \cellcolor{yellow!30}0.76 & 19860.09 & 17030.64 & 15.63 & 0.0385 & 0.0229 & 0.0139 & 0.0017 & 0.602 & 2.577 & 26320.03 & 0.31 & 84079.36 & 42939.27 & 7856.89 & 1365.59 & 3393.84 & 5530.05 & 657.89 & 1367.82 & 83.23 & 4970.99 & 1026.66 & 0.0426 & 0.469 \\
            \hline
            11 & \cellcolor{yellow!30}600 & 143.42 & 293.87 & 25.82 & 35.85 & 35.93 & \cellcolor{yellow!30}0.23 & 7.33 & 1.69 & 5.09 & \cellcolor{yellow!30}9 & \cellcolor{yellow!30}0.10 & \cellcolor{yellow!30}30 & 24986 & \cellcolor{yellow!30}0.80 & 19988.65 & 17103.29 & 15.27 & 0.0328 & 0.0215 & 0.0105 & 0.0008 & 0.501 & 2.375 & 26988.34 & 0.31 & 86067.47 & 44798.82 & 9101.17 & 1514.54 & 3481.72 & 5677.39 & 747.37 & 1514.78 & 88.07 & 5164.91 & 1067.70 & 0.0421 & 0.467 \\
            \hline
            ... & \cellcolor{yellow!30}... & ... & ... & ... & ... & ... & \cellcolor{yellow!30}... & ... & ... & ... & \cellcolor{yellow!30}... & \cellcolor{yellow!30}... & \cellcolor{yellow!30}... & ... & \cellcolor{yellow!30}... & ... & ... & ... & ... & ... & ... & ... & ... & ... & ... & ... & ... & ... & ... & ... & ... & ... & ... & ... & ... & ... & ... & ... & ... \\
            \hline
            198 & \cellcolor{yellow!30}550 & 162.64 & 334.94 & 29.28 & 40.66 & 33.74 & \cellcolor{yellow!30}0.31 & 8.38 & 2.60 & 5.99 & \cellcolor{yellow!30}7 & \cellcolor{yellow!30}0.12 & \cellcolor{yellow!30}35 & 27931 & \cellcolor{yellow!30}0.80 & 22344.88 & 19097.10 & 13.77 & 0.0334 & 0.0210 & 0.0119 & 0.0005 & 0.460 & 2.246 & 31851.04 & 0.36 & 89473.98 & 45849.10 & 8506.47 & 1717.52 & 3631.38 & 6756.21 & 742.30 & 1715.07 & 91.13 & 5436.00 & 1107.67 & 0.0442 & 0.418 \\
            \hline
            199 & \cellcolor{yellow!30}550 & 159.32 & 328.10 & 28.68 & 39.83 & 33.40 & \cellcolor{yellow!30}0.27 & 8.56 & 2.31 & 6.04 & \cellcolor{yellow!30}7 & \cellcolor{yellow!30}0.12 & \cellcolor{yellow!30}20 & 28004 & \cellcolor{yellow!30}0.76 & 22403.04 & 19189.38 & 13.98 & 0.0364 & 0.0210 & 0.0128 & 0.0026 & 0.509 & 2.577 & 28950.81 & 0.33 & 87641.89 & 43958.85 & 7500.36 & 1682.45 & 3551.01 & 6111.38 & 644.61 & 1680.47 & 85.46 & 5249.88 & 1053.85 & 0.0447 & 0.416 \\
            \hline
            200 & \cellcolor{yellow!30}550 & 159.24 & 327.95 & 28.66 & 39.81 & 33.39 & \cellcolor{yellow!30}0.23 & 8.84 & 2.03 & 6.15 & \cellcolor{yellow!30}7 & \cellcolor{yellow!30}0.12 & \cellcolor{yellow!30}20 & 28010 & \cellcolor{yellow!30}0.76 & 22408.17 & 19193.60 & 13.97 & 0.0365 & 0.0210 & 0.0128 & 0.0026 & 0.509 & 2.577 & 28990.48 & 0.33 & 87599.06 & 43910.89 & 7447.19 & 1681.65 & 3549.16 & 6120.17 & 644.46 & 1679.67 & 85.41 & 5250.13 & 1053.50 & 0.0447 & 0.416 \\
            \hline
            201 & \cellcolor{yellow!30}550 & 159.44 & 328.35 & 28.70 & 39.86 & 33.41 & \cellcolor{yellow!30}0.31 & 8.31 & 2.57 & 5.94 & \cellcolor{yellow!30}7 & \cellcolor{yellow!30}0.12 & \cellcolor{yellow!30}20 & 28015 & \cellcolor{yellow!30}0.76 & 22411.85 & 19197.00 & 13.99 & 0.0364 & 0.0210 & 0.0127 & 0.0026 & 0.509 & 2.577 & 28941.62 & 0.33 & 87706.54 & 44014.69 & 7552.67 & 1683.69 & 3553.84 & 6109.34 & 644.84 & 1681.69 & 85.53 & 5251.17 & 1054.46 & 0.0447 & 0.416 \\
            \hline
            202 & \cellcolor{yellow!30}550 & 162.93 & 335.54 & 29.33 & 40.73 & 33.77 & \cellcolor{yellow!30}0.23 & 8.93 & 2.05 & 6.21 & \cellcolor{yellow!30}7 & \cellcolor{yellow!30}0.12 & \cellcolor{yellow!30}35 & 28225 & \cellcolor{yellow!30}0.82 & 22579.84 & 19271.35 & 13.30 & 0.0329 & 0.0209 & 0.0108 & 0.0012 & 0.438 & 2.246 & 32000.52 & 0.36 & 89633.17 & 45773.32 & 8390.20 & 1720.58 & 3638.39 & 6789.55 & 742.93 & 1718.08 & 92.23 & 5552.26 & 1115.30 & 0.0440 & 0.414 \\
            \hline
            203 & \cellcolor{yellow!30}550 & 163.07 & 335.82 & 29.35 & 40.77 & 33.79 & \cellcolor{yellow!30}0.27 & 8.65 & 2.34 & 6.10 & \cellcolor{yellow!30}7 & \cellcolor{yellow!30}0.12 & \cellcolor{yellow!30}35 & 28228 & \cellcolor{yellow!30}0.82 & 22582.64 & 19273.94 & 13.31 & 0.0329 & 0.0209 & 0.0108 & 0.0012 & 0.438 & 2.246 & 31964.25 & 0.36 & 89709.96 & 45847.32 & 8465.67 & 1722.05 & 3641.75 & 6781.46 & 743.23 & 1719.53 & 92.31 & 5553.07 & 1116.00 & 0.0440 & 0.414 \\
            \hline
            204 & \cellcolor{yellow!30}550 & 161.45 & 332.48 & 29.06 & 40.36 & 33.62 & \cellcolor{yellow!30}0.23 & 8.89 & 2.05 & 6.18 & \cellcolor{yellow!30}7 & \cellcolor{yellow!30}0.12 & \cellcolor{yellow!30}30 & 28203 & \cellcolor{yellow!30}0.80 & 22562.64 & 19278.78 & 13.45 & 0.0342 & 0.0209 & 0.0113 & 0.0019 & 0.460 & 2.375 & 30729.39 & 0.35 & 88814.78 & 44972.14 & 7976.56 & 1704.91 & 3602.49 & 6506.34 & 701.37 & 1702.63 & 89.62 & 5454.74 & 1090.96 & 0.0442 & 0.414 \\
            \hline
            205 & \cellcolor{yellow!30}550 & 161.56 & 332.72 & 29.08 & 40.39 & 33.63 & \cellcolor{yellow!30}0.27 & 8.62 & 2.33 & 6.07 & \cellcolor{yellow!30}7 & \cellcolor{yellow!30}0.12 & \cellcolor{yellow!30}30 & 28203 & \cellcolor{yellow!30}0.80 & 22562.66 & 19278.97 & 13.46 & 0.0342 & 0.0209 & 0.0113 & 0.0019 & 0.460 & 2.375 & 30691.60 & 0.35 & 88877.33 & 45034.67 & 8042.14 & 1706.11 & 3605.23 & 6497.94 & 701.60 & 1703.81 & 89.69 & 5455.15 & 1091.51 & 0.0442 & 0.414 \\
            \hline
            206 & \cellcolor{yellow!30}550 & 161.93 & 333.47 & 29.15 & 40.48 & 35.99 & \cellcolor{yellow!30}0.23 & 8.26 & 1.90 & 5.74 & \cellcolor{yellow!30}8 & \cellcolor{yellow!30}0.12 & \cellcolor{yellow!30}30 & 28239 & \cellcolor{yellow!30}0.82 & 22591.52 & 19279.58 & 13.17 & 0.0332 & 0.0206 & 0.0090 & 0.0036 & 0.438 & 2.375 & 28664.73 & 0.32 & 89079.25 & 45207.73 & 8611.28 & 1709.96 & 3614.06 & 6047.99 & 748.65 & 1707.61 & 90.32 & 5549.83 & 1089.87 & 0.0438 & 0.414 \\
            \hline
            207 & \cellcolor{yellow!30}550 & 163.25 & 336.19 & 29.38 & 40.81 & 33.80 & \cellcolor{yellow!30}0.31 & 8.39 & 2.60 & 6.00 & \cellcolor{yellow!30}7 & \cellcolor{yellow!30}0.12 & \cellcolor{yellow!30}35 & 28249 & \cellcolor{yellow!30}0.82 & 22598.90 & 19287.91 & 13.32 & 0.0328 & 0.0209 & 0.0108 & 0.0012 & 0.438 & 2.246 & 31962.71 & 0.36 & 89808.13 & 45929.24 & 8539.41 & 1723.92 & 3646.05 & 6781.11 & 743.62 & 1721.38 & 92.42 & 5555.48 & 1116.99 & 0.0440 & 0.414 \\
            \hline
            208 & \cellcolor{yellow!30}550 & 161.71 & 333.03 & 29.11 & 40.43 & 33.64 & \cellcolor{yellow!30}0.31 & 8.36 & 2.59 & 5.98 & \cellcolor{yellow!30}7 & \cellcolor{yellow!30}0.12 & \cellcolor{yellow!30}30 & 28220 & \cellcolor{yellow!30}0.80 & 22576.00 & 19290.46 & 13.47 & 0.0341 & 0.0209 & 0.0112 & 0.0019 & 0.460 & 2.375 & 30686.49 & 0.35 & 88960.91 & 45104.91 & 8106.28 & 1707.71 & 3608.89 & 6496.80 & 701.92 & 1705.39 & 89.78 & 5457.10 & 1092.33 & 0.0442 & 0.414 \\
            \hline
            209 & \cellcolor{yellow!30}600 & 145.64 & 302.21 & 26.22 & 36.41 & 34.13 & \cellcolor{yellow!30}0.27 & 7.64 & 2.06 & 5.38 & \cellcolor{yellow!30}8 & \cellcolor{yellow!30}0.15 & \cellcolor{yellow!30}30 & 28208 & \cellcolor{yellow!30}0.76 & 22566.33 & 19326.92 & 13.79 & 0.0403 & 0.0225 & 0.0144 & 0.0034 & 0.556 & 2.375 & 29146.86 & 0.33 & 87400.98 & 43554.64 & 7290.91 & 1538.01 & 3540.42 & 6154.83 & 711.67 & 1537.95 & 85.02 & 5264.89 & 1050.12 & 0.0448 & 0.413 \\
            \hline
            ... & \cellcolor{yellow!30}... & ... & ... & ... & ... & ... & \cellcolor{yellow!30}... & ... & ... & ... & \cellcolor{yellow!30}... & \cellcolor{yellow!30}... & \cellcolor{yellow!30}... & ... & \cellcolor{yellow!30}... & ... & ... & ... & ... & ... & ... & ... & ... & ... & ... & ... & ... & ... & ... & ... & ... & ... & ... & ... & ... & ... & ... & ... & ... \\
            \hline
            402 & \cellcolor{yellow!30}600 & 169.12 & 350.93 & 30.44 & 42.28 & 34.41 & \cellcolor{yellow!30}0.23 & 9.07 & 2.09 & 6.31 & \cellcolor{yellow!30}7 & \cellcolor{yellow!30}0.15 & \cellcolor{yellow!30}20 & 42632 & \cellcolor{yellow!30}0.80 & 34105.22 & 29102.81 & 9.08 & 0.0552 & 0.0217 & 0.0124 & 0.0211 & 0.501 & 2.577 & 34704.52 & 0.34 & 101496.52 & 46111.30 & 7536.49 & 1785.97 & 4160.40 & 7394.47 & 663.27 & 1782.60 & 92.44 & 6763.12 & 1126.01 & 0.0533 & 0.274 \\
            \hline
            403 & \cellcolor{yellow!30}600 & 169.28 & 351.25 & 30.47 & 42.32 & 34.42 & \cellcolor{yellow!30}0.31 & 8.52 & 2.64 & 6.10 & \cellcolor{yellow!30}7 & \cellcolor{yellow!30}0.15 & \cellcolor{yellow!30}20 & 42643 & \cellcolor{yellow!30}0.80 & 34114.78 & 29110.90 & 9.09 & 0.0551 & 0.0217 & 0.0123 & 0.0211 & 0.501 & 2.577 & 34630.50 & 0.34 & 101590.79 & 46196.01 & 7630.11 & 1787.62 & 4164.53 & 7377.87 & 663.57 & 1784.23 & 92.53 & 6764.69 & 1126.64 & 0.0533 & 0.274 \\
            \hline
            404 & \cellcolor{yellow!30}650 & 156.04 & 323.78 & 28.09 & 39.01 & 33.05 & \cellcolor{yellow!30}0.27 & 8.49 & 2.29 & 5.98 & \cellcolor{yellow!30}7 & \cellcolor{yellow!30}0.15 & \cellcolor{yellow!30}20 & 43183 & \cellcolor{yellow!30}0.80 & 34546.22 & 29480.78 & 8.92 & 0.0609 & 0.0228 & 0.0145 & 0.0236 & 0.543 & 2.577 & 35932.21 & 0.35 & 101448.27 & 45622.06 & 7079.07 & 1647.80 & 4158.27 & 7670.19 & 638.23 & 1646.28 & 92.23 & 6810.24 & 1126.46 & 0.0536 & 0.271 \\
            \hline
            405 & \cellcolor{yellow!30}650 & 156.12 & 323.95 & 28.10 & 39.03 & 33.06 & \cellcolor{yellow!30}0.31 & 8.23 & 2.55 & 5.89 & \cellcolor{yellow!30}7 & \cellcolor{yellow!30}0.15 & \cellcolor{yellow!30}20 & 43195 & \cellcolor{yellow!30}0.80 & 34555.64 & 29488.78 & 8.92 & 0.0609 & 0.0228 & 0.0145 & 0.0236 & 0.543 & 2.577 & 35908.87 & 0.35 & 101500.34 & 45664.70 & 7122.84 & 1648.64 & 4160.55 & 7664.95 & 638.39 & 1647.11 & 92.28 & 6811.55 & 1126.85 & 0.0536 & 0.271 \\
            \hline
            406 & \cellcolor{yellow!30}650 & 156.01 & 323.72 & 28.08 & 39.00 & 33.05 & \cellcolor{yellow!30}0.23 & 8.76 & 2.02 & 6.09 & \cellcolor{yellow!30}7 & \cellcolor{yellow!30}0.15 & \cellcolor{yellow!30}20 & 43196 & \cellcolor{yellow!30}0.80 & 34556.62 & 29489.73 & 8.91 & 0.0610 & 0.0228 & 0.0146 & 0.0236 & 0.543 & 2.577 & 35998.50 & 0.35 & 101428.14 & 45591.51 & 7035.13 & 1647.48 & 4157.39 & 7685.10 & 638.17 & 1645.96 & 92.22 & 6811.24 & 1126.43 & 0.0536 & 0.271 \\
            \hline
            407 & \cellcolor{yellow!30}600 & 172.48 & 357.90 & 31.05 & 43.12 & 34.75 & \cellcolor{yellow!30}0.27 & 8.86 & 2.39 & 6.25 & \cellcolor{yellow!30}7 & \cellcolor{yellow!30}0.15 & \cellcolor{yellow!30}25 & 44103 & \cellcolor{yellow!30}0.82 & 35282.15 & 30070.24 & 8.70 & 0.0548 & 0.0216 & 0.0116 & 0.0217 & 0.477 & 2.485 & 36044.32 & 0.35 & 103512.72 & 46950.57 & 7901.28 & 1821.42 & 4249.04 & 7695.41 & 693.09 & 1805.27 & 95.13 & 7023.96 & 1151.02 & 0.0539 & 0.265 \\
            \hline
            408 & \cellcolor{yellow!30}600 & 172.41 & 357.75 & 31.03 & 43.10 & 34.74 & \cellcolor{yellow!30}0.23 & 9.15 & 2.10 & 6.36 & \cellcolor{yellow!30}7 & \cellcolor{yellow!30}0.15 & \cellcolor{yellow!30}25 & 44103 & \cellcolor{yellow!30}0.82 & 35282.73 & 30070.80 & 8.70 & 0.0549 & 0.0216 & 0.0117 & 0.0217 & 0.477 & 2.485 & 36101.28 & 0.35 & 103470.25 & 46907.51 & 7848.85 & 1820.68 & 4247.18 & 7708.22 & 692.96 & 1804.53 & 95.10 & 7023.78 & 1150.79 & 0.0539 & 0.265 \\
            \hline
            409 & \cellcolor{yellow!30}600 & 172.64 & 358.23 & 31.08 & 43.16 & 34.76 & \cellcolor{yellow!30}0.31 & 8.60 & 2.66 & 6.15 & \cellcolor{yellow!30}7 & \cellcolor{yellow!30}0.15 & \cellcolor{yellow!30}25 & 44134 & \cellcolor{yellow!30}0.82 & 35307.54 & 30091.85 & 8.71 & 0.0548 & 0.0216 & 0.0116 & 0.0217 & 0.477 & 2.485 & 36036.78 & 0.35 & 103598.57 & 47011.03 & 7955.40 & 1823.12 & 4253.29 & 7693.71 & 693.40 & 1806.96 & 95.21 & 7027.25 & 1151.71 & 0.0540 & 0.265 \\
            \hline
            410 & \cellcolor{yellow!30}650 & 170.93 & 354.69 & 30.77 & 42.73 & 34.59 & \cellcolor{yellow!30}0.27 & 8.83 & 2.38 & 6.22 & \cellcolor{yellow!30}7 & \cellcolor{yellow!30}0.15 & \cellcolor{yellow!30}20 & 52513 & \cellcolor{yellow!30}0.82 & 42010.19 & 35834.59 & 7.52 & 0.0688 & 0.0222 & 0.0131 & 0.0334 & 0.517 & 2.577 & 39243.87 & 0.35 & 111127.69 & 47837.49 & 7773.93 & 1805.10 & 4584.36 & 8417.16 & 666.66 & 1801.48 & 97.38 & 7809.24 & 1178.04 & 0.0593 & 0.223 \\
            \hline
            411 & \cellcolor{yellow!30}650 & 170.89 & 354.59 & 30.76 & 42.72 & 34.59 & \cellcolor{yellow!30}0.23 & 9.11 & 2.10 & 6.34 & \cellcolor{yellow!30}7 & \cellcolor{yellow!30}0.15 & \cellcolor{yellow!30}20 & 52518 & \cellcolor{yellow!30}0.82 & 42014.71 & 35838.61 & 7.51 & 0.0688 & 0.0222 & 0.0132 & 0.0334 & 0.517 & 2.577 & 39313.12 & 0.35 & 111095.77 & 47801.06 & 7724.27 & 1804.59 & 4582.97 & 8432.83 & 666.57 & 1800.97 & 97.35 & 7809.57 & 1177.94 & 0.0593 & 0.223 \\
            \hline
            412 & \cellcolor{yellow!30}650 & 171.04 & 354.90 & 30.79 & 42.76 & 34.60 & \cellcolor{yellow!30}0.31 & 8.56 & 2.65 & 6.13 & \cellcolor{yellow!30}7 & \cellcolor{yellow!30}0.15 & \cellcolor{yellow!30}20 & 52535 & \cellcolor{yellow!30}0.82 & 42027.85 & 35849.57 & 7.52 & 0.0688 & 0.0222 & 0.0131 & 0.0334 & 0.517 & 2.577 & 39221.87 & 0.35 & 111195.23 & 47887.39 & 7823.44 & 1806.20 & 4587.33 & 8412.18 & 666.85 & 1802.56 & 97.43 & 7811.56 & 1178.53 & 0.0593 & 0.223 \\
            \hline
        \end{tabular}
        }
        \caption{Matrice delle Configurazioni: estratto delle configurazioni generate variando i parametri di design. Le righe sono ordinate per block fuel crescente.}
        La configurazione scelta è evidenziata in celeste.
        \label{tab:configurazioni}
    \end{table}
    \vspace*{\fill}
\end{landscape}
  

\noindent Per poter scegliere la configurazione migliore si sono tenute in considerazione le più importanti \textbf{figure di merito}, ovvero parametri fisici dell'aereo come massa massima al decollo, spinta installata ed efficienza aerodinamica, e parametri economici come i costi operativi, il PREE e l'ADP.

Si è notato che il block fuel è un buon indicatore dei costi. Risulta quindi uno dei parametri più importanti per la scelta della configurazione migliore, e quello su cui inevitabilmente si è concentrata l'analisi dei risultati.
\begin{figure}[H]
    \centering
    \includegraphics[width=\textwidth]{E1//Immagini/costi vs block fuel.png}
    \caption{Parametri di costo in funzione del block fuel}
\end{figure}

Un altro parametro direttamente collegato ai costi, apparentemente anche meglio del block fuel, è il peso massimo al decollo.
\begin{figure}[H]
    \centering
    \includegraphics[width=\textwidth]{E1//Immagini/costi vs mtow.png}
    \caption{Parametri di costo in funzione del peso massimo al decollo}
\end{figure}

\noindent In definitiva, le configurazioni con peso al decollo e consumi più bassi sono quelle con costi minori ed efficienza economica maggiore. Si è dunque analizzato il legame tra block fuel e peso massimo al decollo, e si è notato che i due parametri sono legati circa linearmente. Si sono quindi creati degli \textit{scatter plot} per analizzare la dipendenza dalle variabili di design.
\noindent Emergono dei trend che guidano la scelta della configurazione ottimale:

\begin{itemize}
    \item \textbf{Mach di crociera:} È la variabile con l'impatto più chiaro: Mach inferiore comporta valori più bassi di block fuel e peso al decollo. Aumentare la velocità a 0.80 o 0.82 comporta un incremento della resistenza d'onda o, per compensarla, necessita di angoli di freccia elevati che appesantiscono la struttura, portando in ogni caso a un velivolo più pesante e che consuma di più.
    \begin{figure}[H] \centering \includegraphics[width=0.8\textwidth]{E1/Immagini/block fuel = f(WTO, M).png} \caption{Scatter plot di block fuel e MTOW raggruppati per Mach di crociera} \end{figure}
    \item \textbf{Spessore alare ($t/c$):} Le configurazioni con profilo sottile ($10\%$, seguite a breve distanza da quelle al $12\%$) risultano vincenti. Nonostante le ali più sottili siano svantaggiate strutturalmente (la minor resistenza alla flessione comporta un maggior peso della struttura alare), il beneficio aerodinamico in termini di riduzione della resistenza di forma e d'onda è predominante.
    \begin{figure}[H] \centering \includegraphics[width=0.8\textwidth]{E1/Immagini/block fuel = f(WTO, t_c).png} \caption{Scatter plot di block fuel e MTOW raggruppati per spessore del profilo} \end{figure} 
    \item \textbf{Angolo di Freccia ($\Lambda_{25}$):} L'aumento dell'angolo di freccia ha un duplice effetto: da un lato, angoli maggiori ritardano l'insorgere della resistenza d'onda, permettendo efficienze migliori ad alti Mach; dall'altro, una freccia pronunciata incrementa il peso strutturale dell'ala e riduce il $C_{L_{max}}$, penalizzando le prestazioni a bassa velocità. Lo scatter plot mostra che le configurazioni con freccia moderata ($20^\circ - 25^\circ$) tendono a posizionarsi nella regione di minimo consumo e peso, indicando uno svantaggio per gli angoli di freccia estremi.
    \begin{figure}[H] \centering \includegraphics[width=0.8\textwidth]{E1/Immagini/block fuel = f(WTO, freccia).png} \caption{Scatter plot di block fuel e MTOW raggruppati per angolo di freccia} \end{figure} 
    \item \textbf{Allungamento Alare ($AR$):} Il grafico mostra che l'aumento dell'allungamento alare porta ad un miglioramento dell'efficienza, sia in termini di consumi che di peso. Il beneficio aerodinamico dovuto all'alto allungamento prevale sull'aumento di peso strutturale; pertanto, le configurazioni con $AR=9$ risultano le più vantaggiose.
    \begin{figure}[H] \centering \includegraphics[width=0.8\textwidth]{E1/Immagini/block fuel = f(WTO, AR).png} \caption{Scatter plot di block fuel e MTOW raggruppati per aspect ratio} \end{figure} 
    \item \textbf{Carico Alare ($W/S$):} Il carico alare risulta inversamente proporzionale al peso (maggior carico implica minore superficie alare, quindi minor peso strutturale) e direttamente proporzionale al coefficiente di portanza, quindi alla resistenza ($C_{D_i} \propto C_L^2$). Si osserva che le configurazioni peggiori sono quelle a carico più basso, mentre gli effetti positivi e negativi si bilanciano tra i carichi più alti: nonostante le configurazioni a carico alare più alto ($650\,kg/m^2$) abbiano un MTOW inferiore, le configurazioni con minor consumo di carburante e costi operativi diretti sono quelle con carico pari a $600\,kg/m^2$.
    \begin{figure}[H] \centering \includegraphics[width=0.8\textwidth]{E1/Immagini/block fuel = f(WTO, W_S).png} \caption{Scatter plot di block fuel e MTOW raggruppati per carico alare} \end{figure}
    \item \textbf{Rapporto di Rastremazione ($\lambda$):} Questa variabile presenta un comportamento peculiare rispetto alle altre. Come evidenziato dal relativo scatter plot, le curve di tendenza e le nuvole di punti per i diversi valori di $\lambda$ risultano quasi perfettamente sovrapposte, con una leggero vantaggio in termini di peso per le rastremazioni più elevate. Ciò indica che la variazione del rapporto di rastremazione ha un impatto marginale sulle macro-prestazioni del velivolo (MTOW e consumo). Gli effetti contrastanti sul peso strutturale e sull'efficienza aerodinamica (distribuzione di portanza) tendono a bilanciarsi, rendendo la funzione obiettivo poco sensibile a questo parametro.
    \begin{figure}[H] \centering \includegraphics[width=0.8\textwidth]{E1/Immagini/block fuel = f(WTO, lambda).png} \caption{Scatter plot di block fuel e MTOW raggruppati per rapporto di rastremazione} \end{figure}
\end{itemize}

\subsection*{Analisi di un Outlier Significativo}
Di notevole considerazione è un outlier che si distingue per le ottime prestazioni, ovvero la configurazione con $W/S=650\,kg/m^2$, $AR=10$, $M=0.76$, $\Lambda_{25}=20^\circ$, $t/c=10\%$ e $\lambda=0.23$. 
Questa è l'unica configurazione con allungamento alare pari a 10 che rientra nel vincolo geometrico dell'apertura alare massima di 36 metri. Dai risultati, è la configurazione tra tutte con efficienza in crociera maggiore, minimo block fuel, PREE massimo e DOC minimi.
Nonostante le prestazioni superiori, serve valutare l'efficacia di questa soluzione. Data la sua vicinanza ai limiti geometrici (risulta appena 2 cm sotto il limite di apertura), l'elevato allungamento (molto superiore a quello degli aerei esistenti della stessa categoria), la ridotta superficie alare (che compromette le prestazioni a basse velocità), potrebbe essere conveniente optare per una configurazione più conservativa.
\subsection*{Sintesi dei Trend}
Riassumendo, si nota che i valori più bassi di block fuel e MTOW si hanno per: carico alare di 600 o 650 $kg/m^2$, aspect ratio di 9, angolo di freccia di $20\si{\degree}$ o $25\si{\degree}$, spessore del profilo del 10\% e Mach di crociera di 0.76, mentre la sensibilità al rapporto di rastremazione $\lambda$ è molto bassa.
\newpage
\subsection{Scelta della Configurazione}
Sulla base dei risultati ottenuti, la configurazione finale scelta è:
\[ W/S = 600\,kg/m^2, \quad AR=9, \quad M=0.76, \quad \Lambda_{25}=20^\circ, \quad t/c=0.10 \]

Per validare questa scelta, è stato effettuato un confronto diretto con altre configurazioni "ottime":
\begin{enumerate}
    \item \textbf{Minimo DOC:} La configurazione che minimizza i costi operativi diretti in assoluto.
    \item \textbf{Minimo MTOW:} La configurazione più leggera possibile.
    \item \textbf{Minimo Block Fuel:} La configurazione che consuma meno in una missione standard.
    \item \textbf{Miglior Alta Velocità:} La configurazione con $M = 0.82$ che presenta il DOC minore.
\end{enumerate}
\begin{figure}[H]
    \centering
    \includegraphics[width=\linewidth]{E1/Immagini/confronto_ottimizzato.png}
    \caption{Confronto tra la configurazione scelta e gli ottimi globali.}
\end{figure}
Questa analisi evidenzia che:

\begin{itemize}
    \item \textbf{Confronto con Minimo DOC:} Il grafico mostra che la configurazione ottimizzata per il DOC (ottenuta incrementando la velocità a $M=0.80$) offre un vantaggio economico minimo (inferiore allo 0.9\%) rispetto alla scelta effettuata. Tale vantaggio è però controbilanciato da un peggioramento delle prestazioni fisiche: si osserva infatti un aumento sia del peso al decollo (+1.5\%) che del consumo di carburante (+1.5\%), dovuto alla maggiore resistenza d'onda a Mach più elevati.
    
    \item \textbf{Confronto con Minimi Fisici (MTOW e Fuel):} Le configurazioni che minimizzano strettamente il peso (minimo MTOW) e il consumo (minimo block fuel) coincidono con un carico alare più elevato ($W/S=650\,kg/m^2$). Il confronto evidenzia che la configurazione scelta ($W/S=600\,kg/m^2$) paga una penalità estremamente contenuta rispetto a questi ottimi assoluti (differenze dei parametri inferiori all'1\%). Questa minima perdita di "ottimao matematico" è bilanciata e giustificata dal vantaggio di avere un'ala con superficie maggiore, che garantisce un volume interno superiore per il carburante e migliori prestazioni in decollo e atterraggio.
    
    \item \textbf{Confronto con Alta Velocità:} Il confronto con la migliore configurazione veloce ($M = 0.82$) conferma l'impatto negativo dell'aumento di Mach: il prezzo da pagare per la maggiore velocità è un incremento netto del peso (+1.5\%), del consumo (+7.1\%) e una riduzione dell'efficienza PREE (-6.6\%), rendendo questa opzione non competitiva.
\end{itemize}

Per confermare la robustezza della scelta, è stata eseguita un'ulteriore analisi "puntuale", confrontando la configurazione scelta con varianti che differiscono per una singola variabile di design.

\begin{figure}[H]
    \centering
    \includegraphics[width=1\linewidth]{E1/Immagini/Analisi comparativa.png}
    \caption{Confronto tra la configurazione scelta e varianti su singole variabili di design.}
\end{figure}

L'analisi permette di trarre importanti conclusioni sulla sensibilità locale:
\begin{enumerate}
    \item \textbf{Variazione del Carico Alare ($W/S=650\,kg/m^2$ vs $600\,kg/m^2$):} Aumentando solo il carico alare a 650, si ottiene una riduzione marginale del MTOW (-0.6\%) ma un aumento del block fuel (+0.8\%). Questo conferma che il punto di ottimo è sostanzialmente piatto rispetto al carico alare. La scelta di 600, pur leggermente più pesante, è preferibile per garantire margini operativi e volume nei serbatoi.
    
    \item \textbf{Variazione della Freccia ($\Lambda=25^\circ$ vs $20^\circ$):} Aumentare la freccia a $25^\circ$ mantenendo il Mach a 0.76 risulta controproducente. Non vi è alcun vantaggio significativo in termini di resistenza d'onda, poiché a M=0.76 rappresenta meno del 4\% del $C_D$ complessivo, mentre si paga un aumento del peso strutturale che porta ad un incremento del OEW (+1.3\%), mentre la differenza di block fuel rientra nel margine di errore. Questo valida la scelta di una freccia moderata ($20^\circ$) per questo regime di volo.
    
    \item \textbf{Costo della Velocità (Best Fuel $M \ge 0.80$):} Anche selezionando la \textit{migliore} configurazione possibile ad alta velocità (ottimizzata in tutte le altre variabili per minimizzare il consumo), il salto da $M=0.76$ a $M = 0.80$ comporta un aumento del block fuel e del MTOW. Questo dimostra che non esiste una combinazione di parametri capace di rendere efficiente il volo a $M=0.80$ quanto quello a $M=0.76$ per questo specifico velivolo.
\end{enumerate}

In conclusione, la configurazione scelta rappresenta il \textbf{miglior compromesso} tecnico-economico, privilegiando la riduzione dei consumi e dei costi operativi, in linea con le attuali esigenze del mercato aeronautico civile.

\newpage
\section{Descrizione del Velivolo}
Il velivolo risultante dall'ottimizzazione è un bimotore \textit{narrow body} destinato al trasporto di 180 passeggeri su rotte a medio raggio.
La configurazione generale è di tipo convenzionale: ala bassa a sbalzo, impennaggi in coda classici e motori turbofan installati su piloni sotto le ali.

\subsection{Configurazione Interna e Fusoliera}
La fusoliera è stata dimensionata per ospitare 180 passeggeri in classe unica, con una configurazione dei sedili 3+3. Sono previste 4 uscite di emergenza sull'ala, oltre alle porte anteriori e posteriori, per garantire l'evacuazione come da normativa.

\begin{figure}[H]
    \centering
    \includegraphics[width=1\linewidth]{E1/Immagini/domi/250120_Aereo 1_pianta2_CORRETTA.jpg}
    \caption{Layout interno della cabina passeggeri}
\end{figure}

La stiva è suddivisa in compartimenti anteriori e posteriori; le dimensioni permettono di accogliere 5 container standard LD3-45, oltre ai bagagli dei passeggeri.

\begin{figure}[H]
    \centering
    \includegraphics[width=0.9\linewidth]{E1/Immagini/domi/stiva.jpg}
    \caption{Sezione longitudinale della fusoliera}
\end{figure}

\subsection{Geometria Alare}
L'ala ha una pianta trapezoidale con angolo di freccia di $20^\circ$ al 25\% della corda e apertura di 35.56 metri, valore che permette al velivolo di rientrare nei box aeroportuali di classe C. L'elevato allungamento alare ($AR=9$) è stato scelto per massimizzare l'efficienza aerodinamica riducendo la resistenza indotta.

\begin{figure}[H]
    \centering
    \includegraphics[width=0.8\linewidth]{E1/Immagini/domi/250119_Aereo 1_ala_disegno tecnico.jpg}
    \caption{Disegno in pianta della semi-ala con quote principali.}
\end{figure}

Di seguito si alcune viste tridimensionali della geometria alare generata:

\begin{figure}[H]
    \centering
    \begin{minipage}{0.48\textwidth}
        \centering
        \includegraphics[width=\linewidth]{E1/Immagini/domi/250119_Aereo 1_ala_1.jpg}
        \caption{Vista isometrica dell'ala.}
    \end{minipage}
    \hfill
    \begin{minipage}{0.48\textwidth}
        \centering
        \includegraphics[width=\linewidth]{E1/Immagini/domi/250119_Aereo 1_ala_3.jpg}
        \caption{Dettaglio del profilo alare.}
    \end{minipage}
\end{figure}

\subsection{Tabella Riassuntiva}
Di seguito si riportano i parametri fondamentali della configurazione scelta.

\begin{table}[H]
    \centering
    \renewcommand{\arraystretch}{1.3}
    \begin{tabular}{lr}
    \toprule
    \multicolumn{2}{c}{\textbf{Dati Generali e Pesi}} \\
    \midrule
    Massa Massima al Decollo (MTOW) & $84\,323$ kg \\
    Massa Operativa a Vuoto (OEW) & $43\,370$ kg \\
    Block Fuel & $16\,874$ kg \\
    Carico Pagante Max & $22\,415$ kg \\
    \midrule
    \multicolumn{2}{c}{\textbf{Geometria}} \\
    \midrule
    Superficie Alare ($S$) & $140.50 \, m^2$ \\
    Apertura Alare ($b$) & $35.56 \, m$ \\
    Allungamento Alare ($AR$) & 9 \\
    Angolo di Freccia ($\Lambda_{25}$) & $20^\circ$ \\
    Lunghezza Fusoliera & $38.0 \, m$ \\
    Diametro Fusoliera & $4.10 \, m$ \\
    Superficie Impennaggio Orizzontale & $35.12 \, m^2$ \\
    Superficie Impennaggio Verticale & $25.29 \, m^2$ \\
    \midrule
    \multicolumn{2}{c}{\textbf{Propulsione e Prestazioni}} \\
    \midrule
    Spinta al decollo (Totale) & $25\,328 \, kgf \,(248.5\,kN)$ \\
    Rapporto Spinta/Peso ($T/W$) & 0.30 \\
    Mach di Crociera ($M$) & 0.76 \\
    Efficienza in Crociera ($E$) & 15.91 \\
    DOC & $0.0424 \, \$/pax/km$ \\
    \bottomrule
    \end{tabular}
    \caption{Scheda tecnica finale del velivolo.}
\end{table}

\subsection{Diagramma Payload-Range}
La capacità operativa del velivolo è sintetizzata nel diagramma Payload-Range. Il velivolo è in grado di soddisfare pienamente il requisito di missione: alla distanza di progetto di $4\,500$ km, la capacità di carico residua è di $20\,505$ kg, ampiamente superiore ai $16\,740$ kg necessari per trasportare i 180 passeggeri, permettendo di imbarcare anche una quota significativa di cargo ($3\,765$ kg residui) o di operare su tratte ancora più lunghe a pieno carico passeggeri. Il punto di massimo raggio con payload nullo (ferry range) si estende fino a quasi $6\,800$ km, confermando la grande flessibilità operativa.

\begin{figure}[H]
    \centering
    \includegraphics[width=0.85\linewidth]{E1/Immagini/payload_range_finale.png}
    \caption{Diagramma payload-range per la configurazione scelta.}
\end{figure}

\subsection{Stabilità Statica e Dimensionamento dei Piani di Coda}
\subsubsection*{Scelte di design per i piani di coda}
La configurazione dei piani di coda è con architettura tradizionale, con impennaggi posizionati in coda alla fusoliera. Le superfici sono state preliminarmente dimensionate utilizzando dati statistici, rapportandole rispetto alla superficie alare ($S_{ref}$), per poi essere verificate tramite il calcolo dei volumi di coda.

I parametri geometrici definiti per i piani di coda sono i seguenti:
\begin{table}[H]
    \centering
    \renewcommand{\arraystretch}{1.2}
    \begin{tabular}{lccc}
    \toprule
    \textbf{Parametro} & \textbf{Simbolo} & \textbf{Piano Orizzontale} & \textbf{Piano Verticale} \\
    \midrule
    Rapporto di Superficie & $S/S_{ref}$ & $0.25$ & $0.18$ \\
    Superficie & $S_{ht}, S_{v}$ & $35.12 \, m^2$ & $25.29 \, m^2$ \\
    Allungamento Alare & $AR$ & 5 & 1.8 \\
    Angolo di Freccia & $\Lambda_{25}$ & $29^\circ$ & $34^\circ$ \\
    Spessore Relativo & $t/c$ & 0.10 & 0.12 \\
    \bottomrule
    \end{tabular}
    \caption{Caratteristiche geometriche dei piani di coda.}
\end{table}

\subsubsection*{Metodologia di Analisi}
La valutazione della stabilità statica longitudinale e direzionale è stata effettuata utilizzando il metodo dei \textit{volumi di coda}.
La stabilità statica longitudinale richiede che il velivolo, se disturbato da una condizione di equilibrio, generi un momento di beccheggio che tenda a riportarlo alla condizione iniziale. Affinché ciò avvenga, il baricentro del velivolo $X_{CG}$ deve trovarsi davanti al Punto Neutro $X_{N}$, ovvero il centro aerodinamico dell'intero velivolo. La distanza tra questi due punti, normalizzata rispetto alla corda aerodinamica media (MAC), definisce il \textit{margine di stabilità statico} ($SM$).

Le dimensioni dei piani di coda influenzano il valore del  margine di stabilità. I volumi di coda sono parametri adimensionali che quantificano l'efficacia dei piani di coda:
\begin{itemize}
    \item \textbf{Volume di Coda Orizzontale ($V_H$):} Misura l'efficacia del piano orizzontale nel garantire stabilità longitudinale e controllo in beccheggio. Per velivoli da trasporto a getto (\textit{Jet Transport}), i valori tipici sono $V_H \approx 1.1$, con escursioni tipiche tra 0.8 e 1.35.
    \[ V_H = \frac{S_{ht} \cdot l_t}{S_w \cdot \text{MAC}} \]
    Dove $S_{ht}$ è la superficie del piano orizzontale e $l_t$ il suo braccio rispetto al baricentro.
    
    \item \textbf{Volume di Coda Verticale ($V_V$):} Misura l'efficacia del piano verticale per la stabilità direzionale e il controllo dell'imbardata. I valori tipici sono $V_V \approx 0.09$, con range $0.08 \le V_V \le 0.14$.
    \[ V_V = \frac{S_v \cdot l_v}{S_w \cdot b} \]
    Dove $S_v$ è la superficie verticale, $l_v$ la distanza dal baricentro e $b$ l'apertura alare.
\end{itemize}

\subsubsection*{Metodologia di calcolo}
\begin{enumerate}
    \item \textbf{Selezione dei volumi di coda obiettivo:}
    Sulla base dei dati statistici, sono stati selezionati i seguenti coefficienti volumetrici di progetto:
    \begin{itemize}
        \item Volume Coda Orizzontale: $V_{H,des} = 1.0$
        \item Volume Coda Verticale: $V_{V,des} = 0.11$
    \end{itemize}
    Tali valori garantiscono, in prima approssimazione, un margine di stabilità statica longitudinale adeguato e una sufficiente autorità di controllo direzionale.

    \item \textbf{Posizionamento dei Componenti e Stima dei Baricentri:}
    È stata definita una configurazione geometrica longitudinale fissando il bordo d'attacco della radice alare ($X_{LE_{wing}}$) al 42\% della lunghezza fusoliera $L_{fus}$.
    I baricentri delle singole componenti sono stati calcolati analiticamente:
    \begin{itemize}
        \item \textbf{Fusoliera:} include struttura, sistemi, arredi, carrello e payload. Posizionato statisticamente al $46\%$ di $L_{fus}$.
        \item \textbf{Coda ($X_{tail}$):} posizionato al $94\%$ di $L_{fus}$ per massimizzare il braccio di leva.
        \item \textbf{Ala ($X_{wing}$):} calcolato geometricamente in funzione della freccia e della corda:
        \[ X_{wing} = X_{LE_{wing}} + 0.25 \cdot c_{root} + \left(0.35\frac{b}{2} - \frac{d_{fus}}{2}\right) \tan\left(\Lambda_{25}\right) \]
        \item \textbf{Carburante ($X_{fuel}$):} assunto coincidente con il baricentro dell'ala.
        \item \textbf{Motori ($X_{eng}$):} posizionato più avanti rispetto all'ala: $X_{eng} = X_{LE_{wing}} + 0.4 \cdot L_{nacelle}$.
    \end{itemize}

    \item \textbf{Calcolo del Centro di Gravità (CG):}
    La posizione del baricentro totale è stata calcolata sommando i momenti statici dei componenti:
    \[ X_{CG} = \frac{\sum m_i \cdot X_{CG_i}}{\sum m_i} \]
    Esso è stato valutato in due condizioni:
    \begin{itemize}
        \item \textbf{MZFW (Maximum Zero Fuel Weight):} È definito come peso operativo a vuoto più il carico pagante, escludendo quindi il carburante.
        \item \textbf{MTOW (Maximum Take-Off Weight):} Condizione operativa standard a pieno carico.
    \end{itemize}
    La posizione al MTOW è particolarmente critica, in quanto il carburante arretra il baricentro rispetto alla condizione a vuoto, riducendo il margine statico.

    \item \textbf{Verifica dei Volumi di Coda:}
    Sono state calcolate le superfici necessarie per soddisfare i volumi di progetto ($V_{H,des}, V_{V,des}$) alla condizione di MZFW. Successivamente, si è verificato che le superfici effettive del velivolo garantissero coefficienti volumetrici reali ($V_{H,eff}, V_{V,eff}$) all'interno degli intervalli di accettabilità per la condizione di MTOW:
    \[ 0.8 \le V_H \le 1.35 \quad \text{e} \quad 0.08 \le V_V \le 0.14 \]
\end{enumerate}

\subsubsection*{Verifica della Stabilità e Risultati}
L'applicazione della metodologia descritta ha prodotto i risultati riportati in Tabella \ref{tab:risultati_stabilita}.

\begin{table}[H]
    \centering
    \renewcommand{\arraystretch}{1.2}
    \begin{tabular}{lccc}
    \toprule
    \textbf{Parametro} & \textbf{Condizione} & \textbf{Valore Calcolato} & \textbf{Riferimento} \\
    \midrule
    \textbf{Posizione Baricentro ($X_{CG}$)} & MZFW & $18.11$ m (-17.8\% MAC) & \\
                                              & MTOW & $18.63$ m (-7.5\% MAC) & \\
    \midrule
    \textbf{Braccio di Coda ($L_{tail}$)}     & MZFW & $17.61$ m & \\
                                              & MTOW & $17.09$ m & \\
    \midrule
    \textbf{Superficie Richiesta ($V_H=1$)}   & MZFW & $40.29 \, m^2$ & Attuale: $35.12 \, m^2$ \\
    \textbf{Superficie Richiesta ($V_V=0.11$)}& MZFW & $31.21 \, m^2$ & Attuale: $25.29 \, m^2$ \\
    \midrule
    \textbf{Volume Coda Orizzontale ($V_H$)}  & MTOW & \textbf{0.9705} & Target $0.8 - 1.35$ \\
    \textbf{Volume Coda Verticale ($V_V$)}    & MTOW & \textbf{0.1068} & Target $0.08 - 0.14$ \\
    \bottomrule
    \end{tabular}
    \caption{Risultati dell'analisi di stabilità e verifica dei volumi di coda.}
    \label{tab:risultati_stabilita}
\end{table}

\subsubsection*{Conclusioni sulla Stabilità}
Dai risultati emerge che il baricentro del velivolo risulta posizionato davanti alla corda aerodinamica media (valori percentuali negativi), indicando una configurazione naturalmente stabile dal punto di vista statico.
Dai volumi di coda si evince che:
\begin{itemize}
    \item \textbf{Stabilità Longitudinale:} Sebbene la superficie richiesta per ottenere un $V_H=1.0$ al MZFW sia leggermente superiore a quella disponibile ($40.29 \, m^2$ vs $35.12 \, m^2$), il volume di coda effettivo calcolato nella condizione operativa di MTOW è $V_H = 0.97$. Questo valore rientra nell'intervallo di validità ($0.8 \le V_H \le 1.35$), confermando che il piano orizzontale dimensionato statisticamente è sufficiente a garantire la stabilità.
    
    \item \textbf{Stabilità Direzionale:} Analogamente, il volume di coda verticale al MTOW risulta $V_V = 0.107$. Anche questo valore soddisfa largamente il criterio di progetto ($0.08 \le V_V \le 0.14$), assicurando adeguate caratteristiche di stabilità direzionale.
\end{itemize}

L'escursione del baricentro tra vuoto e pieno carico (arretramento del 10.3\% della MAC) è contenuta e gestibile, in partricolare grazie al fatto che resti davanti alla corda aerodinamica media; ciò conferma la bontà del layout generale del velivolo.

In conclusione, la configurazione geometrica scelta e il posizionamento dei componenti portano a un velivolo stabile ed equilibrato, senza richiedere modifiche sostanziali alle superfici degli impennaggi definite in fase preliminare.

\begin{figure}[H]
    \centering
    \includegraphics[width=0.9\linewidth]{E1/Immagini/Escursione Baricentro.png}
    \caption{Visualizzazione dell'escursione del baricentro rispetto alla MAC e alla posizione della coda.}
\end{figure}